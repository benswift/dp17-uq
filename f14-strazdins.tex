% a list of your ten career-best research outputs, with a brief
% paragraph for each research output explaining its significance (four
% pages maximum).
% Provide the full reference for each, 
% provide information on any relevant ARC grant scheme that your were CI on,
% add a brief paragraph for each, explaining and justifying its impact,
% mark with an asterisk the research outputs relevant to this proposal

\documentclass[a4paper,oneside,12pt]{article}
\usepackage{arcdoc}

\begin{document}

\subsubsection*{G12~ Ten Career-best Publications: PE Strazdins}

Citation counts as at 4 November 2015 from Google Scholar.

\setlength{\leftmargini}{8mm}
~ \\
\textbf{Scholarly Book Chapters:}
\begin{enumerate}
\item * {Peter Strazdins}, K.\ Homma, K.\ Nagase, V.\ Nguyen, M.\
      Noro, and T.\ Yamagajo,
{\em Efficient {P}arallel {E}lectromagnetic {F}ield {A}nalysis
      {E}xploiting {S}ymmetry},
	chapter in {\em Practical Applications of Parallel Computing},
      Advances in Computation: Theory and Practice Series Volume 12,
      pp.\ 201--222. NOVA Science Publishers Inc, Huntington NY,
      2002. 

Describes how the first parallelization of symmetric
indefinite solvers and other techniques were used to speedup an already
efficiently parallelized commercial application (ACCUFIELD) by a factor
of three. 

\end{enumerate}
~\\[-2mm]
\textbf{Refereed Journal Articles:}
\begin{enumerate}
\setcounter{enumi}{1}
%\setlength{\itemsep}{1mm}
%\setlength{\itemsep}{0.5ex plus0.2ex}

\item * {Peter Strazdins}, 
	{\em A Comparison of Lookahead and {A}lgorithmic {B}locking
      Techniques for {P}arallel {M}atrix {F}actorization.} 
	International Journal of Parallel and Distributed Systems and
      Networks, 4(1), pp.\ 26--35, April 2001. 

First comparison of leading load balancing techniques for parallel
dense linear algebra, also developing formal properties and
performance models. The technique is of continuing relevance, and in
particular for manycore devices such as GPUs. 44 citations, 25 over
the last 5 years. The conference version of this paper (1998) has received
12 citations, making 56 in total.


\item P.E. Strazdins.
                {\em Accelerated methods for performing the {LDLT} decompositio\
n.}
                {ANZIAM J.} 42(E), pages C1328--C1355, Dec 2000.

First ever distributed memory parallelization of the LDLT
decomposition.  Proposed and evaluated several algorithms along a
spectrum of performance and numerical stability tradeoffs. Led to a
follow-up paper co-authored by John Lewis from Boeing, a world leader
in the area of numerical algorithms. Cited by 10.

\item R.P. Brent and P.E. Strazdins. {\em Implementation of BLAS Level
  3 and LINPACK Benchmark on the AP1000}, Fujitsu Scientific and
  Technical Journal, 29(1):61–70, 1993.  A collaboration with a
  world-leading computer scientist leading to the first description of
  what later became known as the algorithmic blocking technique for
  matrix factorization; also one of the first papers on parallel BLAS
  implementation. 23 citations.

\end{enumerate}

~\\[-2mm]
\textbf{Refereed Conference Papers:}
\begin{enumerate}
\setcounter{enumi}{4}
%\setlength{\itemsep}{1mm}
%\setlength{\itemsep}{0.5ex plus0.2ex}

\item

* Md Mohsin Ali, Peter E. Strazdins, Brendan Harding, Markus Hegland,
J. Walter Larson, {\em A Fault-Tolerant Gyrokinetic Plasma Application          
using the Sparse Grid Combination Technique}, Proceedings of the 2015
International Conference on High Performance Computing \& Simulation
(HPCS 2015), pp499-507, Amsterdam, July 2015.

Culminating paper of a series of papers produced by
LP100200079. Demonstrated that an existing, widely used and highly
complex real-world application running over thousands of processes can
be made robustly tolerant to hard faults. The novelty lies in using a form
of Algorithmic-Based Fault Tolerance to do so (the project was the
first to use it for that purpose).  One of CI Strazdins contributions
was solely devising and implementing the first-ever distributed memory
SGCT algorithm -- a major achievement in its own right.
The technique is applicable to a large
class of scientific simulations, and the project has similarly made
two other major applications fault-tolerant since. 
In recognition of its breakthrough research, it was given the {\bf
  Outstanding Paper Award} at HPCS-15.

Under grant LP100200079.


\item * Rui Yang, Joseph Antony, Alistair Rendell, Danny Robson and
  Peter Strazdins, {\em Profiling Directed NUMA Optimization on Linux
    Systems: A Case Study of the Gaussian Computational Chemistry
    Code}, Proceedings of the 2011 IEEE International Parallel \&
  Distributed Processing Symposium, Anchorage, p1033-1042, May 2011.
  
  ERA Rank A venue, regarded as one of the best in the field. Used
  novel performance analysis techniques and computer simulation tools
  to reveal deep aspects of the performance of NUMA systems.  9
  citations.

 Under grant LP0774896.


\item Jie Cai, Peter E. Strazdins, and Alistair P. Rendell, {\em
Region-Based Prefetch Techniques for Software Distributed Shared Memory
Systems}, Proceedings of the  2010 10th IEEE/ACM International
Conference on Cluster, Cloud and Grid Computing, pages 113-122, IEEE,
Melbourne, May 2010.

A new technique significantly improving the accuracy of prefetch
prediction over earlier work with the potential to make shared memory
programming on clusters more widely applicable. ERA Rank A venue. 

Under grant LP0669726.

\item * Andrew Over, Bill Clarke and Peter Strazdins, 
{\em A Comparison of Two Approaches to Parallel Simulation of
      Multiprocessors.},
      Proceedings of the IEEE International Symposium on Performance Analysis of
      Systems and Software (ISPASS'07), San Jose, April 2007.

First paper on parallel computer simulation quantifying stability issues
and accuracy-performance tradeoffs; will become increasingly more
important in the coming `manycore' era. 13 citations. Venue's citation
rate of 4.8 in 2007.  

Under grant LP0347178.

\item  Peter Strazdins and John Uhlmann,
{\em A Comparison of Local and Gang Scheduling on a Beowulf
      Cluster}, Proceedings of the IEEE International Conference of Cluster
      Computing, pp.\ 55--62, San Diego, Sep 2004. 

First paper to challenge the need for gang scheduling, the
then-dominant scheduling paradigm for interactive clusters. 
Provided analysis as well as compelling experimental results.
ERA rank A venue, regarded as a leading venue in the field. 27 citations.


\item * Andrew Over, Peter Strazdins and Bill Clarke, {\em Cycle
  Accurate Memory Modelling: A Case-Study in Validation.}, Proceedings
  of the IEEE International Symposium on Modeling, Analysis, and
  Simulation (MASCOTS'05), pp.\ 85--94, Atlanta, September 2005.

One of the first papers examining validation issues on a real parallel
machine, exposing greater difficulties than formerly understood.
Will become increasingly important in the manycore era. 
ERA Rank A venue. 7 citations.

Under grant LP0347178.

\end{enumerate}

%%%%%%%%%%%%%%%%%%%%%%%%%%%%%%%%%%%%%%%%%%%%%%%%%%%%%%%%%%%%%%%%%%%%%%

\end{document}
\item * Peter Strazdins,
{\em Optimal {L}oad {B}alancing {T}echniques for
      {B}lock-{C}yclic {D}ecompositions for {M}atrix {F}actorization},
         In 2nd International Conference on Parallel and
      Distributed Computing and Networks (PDCN'98), pp 192--199, IASTED,
      Brisbane, December 1998.

Proposes, analyzes and evaluates a new technique in parallel dense
linear algebra; key finding that small storage block sizes are optimal
for this and related techniques. Cited by 6.


\item Jie Cai and Peter E. Strazdins,
{\em An Accurate Prefetch Technique for                                         
Dynamic Paging Behavior for Software Distributed Shared Memory},
41st International Conference on Parallel Processing (ICPP),
pp.209-218, Sep 2012.

  The paper develops prefetching techniques which brought a a dramatic
  improvement in both coverage and efficiency over the work of other
  authors. ERA rank A venue. Under grant LP0669726.

