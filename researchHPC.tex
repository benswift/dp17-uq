\subsubsection*{High-Performance Computing Systems Support}

A reliance on high performance computing for the evaluation of
scientific models provides additional challenges to the technical side
of the project. Specifically, our algorithms must be highly scalable
and robust to errors and faults in the computer system layer. There
have been many recent developments in both highly-scalable algorithms
for the sparse grid combination technique \cite{sgctalg15,pdsec15extsgctalg} 
which will be of use for
this and, additionally, it has been recently shown that such
computations can be made
robust~\parencite{HardingHLS2015,AliEtal2015,Ali11022016}. By leveraging and
continuing to develop these algorithms we can ensure that the
offline components of our software system that require high performance
computing resources will be both scalable and robust.

In this project we will leverage on-demand compute resources, such as
the Amazon AWS cloud~\parencite{amazonAws} and the National Compute
Infrastructure NCI Cloud~\parencite{nciCloud}. Using these cloud
services will further improve the project's ability to deliver timely
results in high-pressure and time-critical decision making scenarios.

Once again, we emphasise that our approach to live coding of test
software is a novel aspect of our methodology. The \emph{Extempore}
tool that we have developed has been shown to be able to harness and
steer scientific simulation in real time. In this part of the project,
we will apply this tool to the real-time steering of the computer
systems layer itself.

This will involve developing code libraries to assist the developer in
the live performance evaluation and tuning of these complex and highly
parallel simulations. For example, groups of processes will be
allocated to different parts of the simulation. If some parts are
delayed relative to others, processes can be `stolen' from the faster
group to improve load balance \cite{parSGCT16}. Communication
bottlenecks can be identified and alternate communication algorithms
can be employed to rectify this.  Different computational kernels can
be selected depending on the current memory system and floating point
performance.  It should be noted that this fine-tuning is not only
application-dependent, but within an application, it depends on the
workload selected, and even within that, may depend on the current
phase of the simulation. Only Live Programming by \emph{Extempore} has
the flexibility and agility to facilitate such a degree of performance
tuning.
