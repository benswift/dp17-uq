\subsubsection*{High-Performance Computing Systems Support}

A reliance on high performance computing for the evaluation of
scientific models provides additional challenges to the technical side
of the project. Specifically, our algorithms must be highly scalable
and robust to errors and faults in the computer system layer. There
have been many recent developments in both highly-scalable algorithms
and the sparse grid combination technique which will be of use for
this and, additionally, it has been recently shown that such
computations can be made
robust~\parencite{HardingHLS2015,AliEtal2015,Ali11022016}. By leveraging and
continuing to develop these algorithms we can ensure that the
offline components of our software system that require high performance
computing resources will be both scalable and robust.

In this project we will leverage on-demand compute resources, such as
the Amazon AWS cloud~\parencite{amazonAws} and the National Compute
Infrastructure NCI Cloud~\parencite{nciCloud}. Using these cloud
services will further improve the project's ability to deliver timely
results in high-pressure and time-critical decision making scenarios.

Once again, we emphasise that our approach to live-coding of test
software is a novel aspect of our methodology. The \emph{Extempore}
tool that we have developed has been shown to be able to harness and
steer scientific simulation in real time. In this part of the project,
we will apply this tool to the real-time steering of the computer
systems layer itself.


