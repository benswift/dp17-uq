\noindent{\bf Framework/Advancing Knowledge -- AIM 3:}
A reliance on high performance computing for the evaluation of
scientific models provides additional challenges to the technical side
of the project. Specifically, our algorithms must be highly scalable
and robust to errors and faults in the computer system layer. 
We will take advantage of 
recent developments in both highly-scalable algorithms
for the sparse grid combination
technique \cite{sgctalg15,pdsec15extsgctalg} and other results that demonstrate that such
computations can be made
robust~\parencite{HardingHLS2015,AliEtal2015,Ali11022016}. We will then apply live-coding to the real-time steering of the computer
systems layer itself. This will involve developing code libraries to assist the developer in
real-time performance-evaluation and tuning of these complex and highly
parallel simulations. For example, groups of processes will be
allocated to different parts of the simulation. If some parts are
delayed relative to others, processes can be `stolen' from the faster
group to improve load balance \cite{parSGCT16}. Communication
bottlenecks can be identified and alternate communication algorithms
can be employed to rectify this.  Different computational kernels can
be selected depending on the current memory system and floating point
performance.  It should be noted that this fine-tuning is not only
application-dependent, but within an application it depends on the
workload selected on the current
phase of the simulation. Of live-coding systems, only \emph{Extempore} has
the flexibility and agility to facilitate such performance
tuning.

%By
%leveraging and continuing to develop these algorithms we can ensure
%that the offline components of our software system that require high
%performance computing resources will be both scalable and robust.

\iffalse
In this project we will leverage on-demand compute resources, such as
the Amazon AWS cloud~\parencite{amazonAws} and the National Compute
Infrastructure NCI Cloud~\parencite{nciCloud}. Using these cloud
services will further improve the project's ability to deliver timely
results in high-pressure and time-critical decision making scenarios.

Once again, we emphasise that our approach to live coding of test
software is a novel aspect of our methodology. 
\fi




Live coding of the systems layer of our project will
allow rapid prototyping and performance-tuning of our software so that
it can be run on realistic (cluster and cloud) HPC computer
support. Most importantly, live-coding will allow a computer support
expert to optimise the execution of running simulations under
conditions where the computer systems may be unreliable or subject to
rapid change (as might happen in some disaster-management scenarios).\\
