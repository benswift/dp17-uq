\iffalse
\subsection*{ROLE OF PERSONNEL}
\fi
\subsection*{INVESTIGATORS}

% - Summarise the role, responsibilities and contributions of each Chief Investigator (CI) and Partner Investigator (PI)
% - Summarise the roles and levels of involvement of other Participants, for example, technical staff, Research Associates and other personnel
% - Describe how each Participant will ensure they have the ‘time and capacity’ to undertake the proposed research, taking into account any other grants or roles that they hold.

%The personnel on this grant cover the key research areas discussed in
%the \emph{Research Project} section of this application:
%
%\begin{itemize}
%\item \textbf{Steve Roberts}: uncertainty quantification, sparse grids, flood
%  modelling
%\item \textbf{Markus Hegland}: sparse grids
%\item \textbf{Peter Strazdins}: sparse grids, distributed computing
%\item \textbf{Henry Gardner}: HCI, interfaces
%\end{itemize}


The personnel involved in this project will be CIs Gardner, Strazdins,
Roberts and Hegland, two Post-doctoral Research Associates and four
PhD students. This project is split equally between the ANU Research
School of Computer Science and the ANU Mathematical Sciences
Institute. The four CIs are senior academics with many years of
experience in running projects and many years of experience working
with each other. Although CI Gardner is the lead CI on this grant
application, the management structure of the project will be flat and
deeply collaborative.


{\bf CI Gardner} will be responsible for the overall project. He is a senior
academic with many years of experience in computational science, high
performance computing, virtual reality, Human Computer Interaction and
computer music. He was director and head of school of ANU Computer
Science from 2008 to 2013 and is presently the Associate Dean of
Higher Degree Research in the ANU College of Engineering and Computer
Science. In recent years, CI Gardner has been leading a research group
that has developed the live coding ``Extempore'' programming
system. This system originated in the creative arts to support live
computer music, but has now been shown to be able to harness and
interactive with traditional computational science applications in
real time with negligible performance overhead. CI Gardner has a 40\%
research allocation and his term as Associate Dean will expire in 2018
providing him with opportunity to provide the proposed 20\% allocation
to the present project.
%
{\bf CI Strazdins} will be responsible for the high performance and
distributed computing components of the project, based on his
expertise in high performance computing, where he has a number of
significant and well cited publications
~\parencite{AliEtal2015,StrazdinsEtal2015,Ali11022016}.  CI Strazdins
has a 40\% research allocation, leaving sufficient research time to
cover the proposed 15\% allocation to this project.
%
{\bf CI Roberts} will lead the inundation modelling, and uncertainty
quantification components, based on his expertise in computational
fluid dynamics and the use of sparse grid based uncertainty
quantification~\parencite{deBaarRDM2015,JakemanRoberts2013,anugamanual,nielsen2005hydrodynamic}.
CI Roberts has a 40\% research allocation, with no current grant
obligations, leaving sufficient research time to cover the proposed
20\% allocation to this project.
%
{\bf CI Hegland} will be responsible for the sparse grid and reduced
model component of the project, based on his extensive expertise in
computational mathematics, in particular his expertise in the analysis
of the combination technique for sparse
grids~\parencite{AliEtal2015,HardingHLS2015,Ali11022016}.  CI Hegland
has a 60\% research allocation. In 2018 will he contribute 20\% of his
research time to the DP150102345, and potentially a 20\% contribution
to LP160100624 leaving sufficient research time to cover the proposed
20\% allocation to this project.


Our project is strengthened by a collaboration we are building with Dr
Bert Debusschere and Dr John Jakeman both from Sandia National
Laboratories, USA. They both have very strong records in uncertainty
quantification research.  Dr Jakeman has expertise on uncertainty
quantification techniques based on his extensive experience in using
adaptive sparse grids, having applied such techniques to complex
modelling problems~\parencite{JakemanRoberts2013,jakemanNumerical2010,
  Jakeman2015} and has already collaborated with CI Roberts (his
former PhD supervisor)~\parencite{JakemanRoberts2013}.  Dr Debusschere
has extensive expertise in uncertainty quantification through to fault
tolerant numerical solvers, having already collaborated with CI
Strazdins~\parencite{parSGCT16}.  

\iffalse
Both Dr Jakeman and Dr Debusschere
have offered to host visits from our group to the Livermore and
Albuquerque Sandia Laboratories. We plan to support reciprocal visits
to ANU.
\fi

{\bf One post-doctoral fellow}, to be situated in the Research School
of Computer Science (RSCS), will develop large portions of the live
programming tools and software interface which will interact with the
scientific models and will conduct experiments on live scenarios where
human actors mimic the interactions and information flows involved in
the group dynamics of disaster response with simulation support. Due
to the large skill set and responsibilities of this position it has
been factored into the budget at level B3.  One possible candidate for
this position is Dr Ben Swift who is currently a post doc in the RSCS
and has work extensively in HCI and has collaborated extensively with
CI Gardner, on work relevant to this project
~\parencite{martin2015tracking,martin2016intelligent,swiftLive2016,
  swift2013visual,swift2014coding}.

{\bf A second post-doctoral fellow}, to be situated in the
Mathematical Sciences Institute (MSI), will need to have a very
different skills set to the computer science research fellow. This
research fellow will develop the numerical methods for computing
sparse grid surrogates and reduced basis models which will be used to
efficiently propagate uncertainties in model applications of tsunami
and flood surge events. This position is budgeted at level B1 (raising
to B2 in year 3).  One possible candidate for this position is the
current PhD student of CI Hegland, Mr Brendan Harding, who has has
collaborated with CIs Hegland, Strazdins and Roberts, all on work
relevant to this project~\parencite{Ali11022016, AliEtal2015,
  deBaarHarding2015, HardingHLS2015, StrazdinsEtal2015}.


We are requesting ARC funding for {\bf four PhD students}, and we anticipate
obtaining funding for another two PhDs from Australian Postgraduate
Awards. Even though we have designated these PhD students to either
computer science or mathematics, the projects will be
inter-disciplinary, with all four CIs being on the supervisory panels
of each of the students. The four PhD students will be split between
the two departments—two in RSCS and two in MSI, focusing on specific
aspects of the project, essentially aligned with the four CIs areas of
responsibility. In RSCS, these will be the dynamic distributed
computing infrastructure and the development of live interfaces for
data visualisation. In the MSI, these will be the uncertainty
quantification and the development of numerical methods for the sparse
grid technique, and the reduced basis method, applied to our
inundation model example. 

\iffalse
We anticipate being able to support several
Honours projects in computer science and mathematics during this
project.  Some of these projects will relate to the core mission of
the project detailed above, while others will concentrate on related
studies with a view to achieving high-impact pure and applied research
and training young people for a career in the computing and
mathematical modelling disciplines
\fi

\iffalse
We anticipate being able to support a number of Honours projects in
computer science and mathematics during the course of this
project. Some of these projects will relate to the core mission of the
project detailed above, others will concentrate on related studies
with a view to achieving high-impact pure and applied research and
training young people for a career in the computing and mathematical
modelling disciplines.
\fi

