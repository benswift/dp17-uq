
\subsection*{ROLE OF PERSONNEL}
% - Summarise the role, responsibilities and contributions of each Chief Investigator (CI) and Partner Investigator (PI)
% - Summarise the roles and levels of involvement of other Participants, for example, technical staff, Research Associates and other personnel
% - Describe how each Participant will ensure they have the ‘time and capacity’ to undertake the proposed research, taking into account any other grants or roles that they hold.

%The personnel on this grant cover the key research areas discussed in
%the \emph{Research Project} section of this application:
%
%\begin{itemize}
%\item \textbf{Steve Roberts}: uncertainty quantification, sparse grids, flood
%  modelling
%\item \textbf{Markus Hegland}: sparse grids
%\item \textbf{Peter Strazdins}: sparse grids, distributed computing
%\item \textbf{Henry Gardner}: HCI, interfaces
%\end{itemize}

The personnel involved in this project will be the CIs Gardner,
Roberts, Strazdins and Hegland, 
%the PIs Jakeman and Debusschere, 
and
two Postdoctoral Research Associates and four PhD students.  This
project is split between the ANU Research School of Computer Science
(CI Gardner, CI Strazdins, one PDRA, two PhD students) the ANU
Mathematical Sciences Institute (CI Roberts, CI Hegland, one PDRA and
two PhD students).
%together will support from Sandia National Labs (PI
%Jakeman and PI Debusschere). 


CI Strazdins will be
responsible for the high performance and distributed computing
components of the project, based on his expertise in high performance
computing~\parencite{AliEtal2015,StrazdinsEtal2015,Ali11022016}.  
CI Strazdins has a 40\% research allocation. In 2017 he contribute of 20\% of his research time to the LE150100030, leaving sufficient research time to cover the proposed 20\% allocation to this project. 
%
CI Gardner will be responsible for the interactive computing and visual interface components of this project, based on
his expertise in HCI and visualisation~\parencite{martin2016intelligent,martin2015tracking,swift2014coding,swift2013visual,sorensen2010programming,swiftLive2016}.  CI Gardner has a 40\% research allocation. In 2017 he will have a potential contribution of 10\% of his research time to the LP160100624 which is presently under consideration, leaving sufficient research time to cover the proposed 20\% allocation to this project. 
%
CI Roberts will lead the inundation modelling, and
uncertainty quantification components, based on his expertise in
computational fluid dynamics and the use of sparse grid based
uncertainty quantification~\parencite{deBaarRDM2015,JakemanRoberts2013,anugamanual,nielsen2005hydrodynamic}.  
CI Roberts has a 40\% research allocation, with no current grant obligations,  leaving sufficient research time to cover the proposed 20\% allocation to this project. 
%
CI Hegland will be responsible for the
sparse grid and reduced model component of the project, based on his
extensive expertise in computational mathematics, in particular his
expertise in the analysis of the combination technique for sparse
grids~\parencite{AliEtal2015,HardingHLS2015,Ali11022016}. 
CI Hegland has a 60\% research allocation. In 2017 he contribute of 20\% of his research time to the DP150102345, and potentially 20\% contribution to LP160100624 leaving sufficient research time to cover the proposed 20\% allocation to this project. 

%PI Jakeman will provide expertise on uncertainty
%quantification techniques based on his extensive experience in
%applying such techniques to complex modelling 
%problems~\parencite{JakemanRoberts2013,jakemanNumerical2010,Jakeman2015}.  
%PI Debusschere will provide his extensive expertise in uncertainty
%quantification through to fault tolerant numerical solvers.

One post-doctoral fellow, to be situated in the
Research School of Computer Science (RSCS), will develop large
portions of the live programming tools and software interface which
will interact with the scientific models and will conduct experiments
on live scenarios where human actors mimic the interactions and
information flows involved in the group dynamics of disaster response
with simulation support. Due to the large skill set and responsibilities of 
this position it has been factored into the budget at level B3.
One possible candidate for this position is 
Dr Ben Swift who is currently a post doc in the RSCS and has work 
extensively in HCI and has collaborated extensively with CI Gardner, on work 
relevant to this project ~\parencite{martin2015tracking,martin2016intelligent,swiftLive2016,
swift2013visual,swift2014coding}.

The second post-doctoral fellow, to be
situated in the Mathematical Sciences Institute (MSI), 
will need to have a very different skills set to the computer 
science research fellow. This research fellow will develop
the numerical methods for computing sparse grid surrogates and reduced
basis models which will be used to efficiently propagate 
uncertainties in
model applications of tsunami and flood surge events. This 
position is budgeted at level B1 ( raising to B2 in year 3).
One possible candidate for this position is the current PhD 
student of CI Hegland, Mr Brendan Harding,  who has just 
submitted his PhD thesis on ``Fault Tolerant Computation of
Hyperbolic PDEs with the Sparse Grid Combination Technique''.
Mr Harding has collaborated with his supervisor, 
CI Hegland, along with CI Strazdins and CI Roberts, all
on work relevant to this project~\parencite{Ali11022016,AliEtal2015,deBaarHarding2015,
HardingHLS2015,StrazdinsEtal2015}.


The four PhD students will be split between the two departments---two
in RSCS and two in MSI, focusing on specific aspects of the project,
essentially aligned with the four CIs areas of responsibility.  In
RSCS, these will be the dynamic distributed computing infrastructure
and the development of live interfaces for data visualisation. In the
MSI, these will be the uncertainty quantification and the development
of numerical methods for the sparse grid technique, and the reduced
basis method, applied to our inundation model example.
We are requesting ARC funding for 2 PhDs, with funding for the other two PhDs coming 
from  Australian Postgraduate Awards.  Even though we have designated these 
PhD students as either being computer science or mathematics, we envisage that the 
projects will be inter-disciplinary, with all four CIs being on the supervisory panels of 
each of the students. 

We anticipate being able to support a number of Honours projects in
computer science and mathematics during the course of this
project. Some of these projects will relate to the core mission of the
project detailed above, others will concentrate on related studies
with a view to achieving high-impact pure and applied research and
training young people for a career in the computing and mathematical
modelling disciplines.

