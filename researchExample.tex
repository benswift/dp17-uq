

\subsubsection*{A specific storm-surge example}
Consider the problem of predicting the maximum 
storm surge water level at a location (or locations) along a coast threatened by a cyclone. We will denote the geographical area of interest by $\Omega$ 
(the region threatened by the cyclone) and consider only 
spatial points $(x,y) \in \Omega$. We will also restrict our interest to times $t$ to an interval of time $T = [t_0, t_1]$ (the duration of the storm). 
A standard mathematical model for storm surge is provided 
by the shallow water wave equation,  which encodes the conservation of water and Newton's second Law (rate of change of momentum - Force = 0) 
\begin{subequations}
 \label{eqn:storm} 
\begin{align}
\frac{\partial h}{\partial t} +
\frac{\partial }{\partial x} \left(uh\right) + \frac{\partial }{\partial y} \left(vh\right) -R & =  0 ,\\
\frac{\partial uh}{\partial t} +
\frac{\partial }{\partial x} \left(u^2h +  \frac12 gh^2 \right) 
+ \frac{\partial }{\partial y} \left(vuh\right) + gh \frac{\partial b}{\partial x} 
- \frac{h}{\rho} \frac{\partial P}{\partial  x} - S_{fx} - S_{wx}  &= 0 ,   \\
\frac{\partial vh}{\partial t} +
\frac{\partial }{\partial x} \left(uvh  \right) 
+ \frac{\partial }{\partial y} \left(v^2h + \frac12 g h^2 \right) + gh \frac{\partial b}{\partial y} 
- \frac{h}{\rho} \frac{\partial P}{\partial  y} - S_{fy} -  S_{wy}& = 0,  
\end{align}
\end{subequations}
where $h(x,y,t)$ is the depth of water, $u(x,y,t)$ and $v(x,y,t)$ are the $x$ and $y$ horizontal components of water velocity. In particular $h$, $u$ and $v$ are functions defined on $\Omega \times T = \mathcal{D}$. 

Here $g$ is the gravitational constant (9.81) and $\rho$ is the density of water. 
The other terms constitute input data to the model. 
The bathymetry $b(x,y)$, is the  elevation of the ocean bed.
The rate of rainfall on the region over time is $R$.
The atmospheric pressure is given by $P$ (which generates a surge due to spatial differences in atmospheric pressure associated with a large cyclone).
The $x$ and $y$ components of the frictional force generated by the flow of the surge over the ocean bed (flow over sand is different than flow through mangroves) is given by $S_{fx}$ and $S_{fy}$, respectively. 
The $x$ and $y$ components of the surface stress force (generated by the wind) is given by $S_{wx}$ and $S_{wy}$, respectively.
As is evident, there are many opportunities for uncertainty in the input  data defining this model. 

In our papers~\parencite{anugamanual,nielsen2005hydrodynamic}  these
equations and their approximation using the  AnuGA package is shown to provide a reliable
model of general flows associated with inundation due storm-surge as well as riverine flooding and tsunamis.

The parameter space $\mathcal{P}$ will represent the possible variation in the input data $R$, $b$, $P$, $S_f$, $S_w$, as well as design parameters describing actions such as raising or lowering flood barriers and releasing or diverting flow from upstream rivers, or flood basins, or indeed constructing emergency levees. 
The model solution 
$$
U_{\mathbf{p}} (\mathbf{x})  = (  h(x,y,t) , u(x,y,t) ,  v(x,y,t) )
$$
represents the water depth and velocity fields obtained by solving the model problem for a choice of parameter $\mathbf{p}$ at a particular location and time $\mathbf{x} = (x,y,t)$.

The quantity of interest $Q(U_{\mathbf{p}})$ in this case will be the maximum storm surge height at a particular location $(x_0, y_0) \in \Omega$,
$$ 
Q(U_{\mathbf{p}})  = \max_{t_0 \leq t \leq t_1} \left( h(x_0,y_0,t) + b(x_0,y_0) \right).
$$
The aim of  uncertainty quantification is to obtain useful relationships between the variations in pressure, wind and rainfall (changes of $\mathbf{p} \in \mathcal{P}$) and $Q(U_{\mathbf{p}})$, including identifying which components of the inputs  have the greatest influence on the result. 

A completely general parameter space $\mathcal{P}$ may lead to an intractable problem. It is sensible to look for a lower dimensional manifold $\mathcal{C} \subset \mathcal{P}$. This can be done algorithmically. Or in this case $\mathcal{C}$ might represent a model of the pressure, wind and rainfall associated with a cyclone of specific intensity and location. This would still involve uncertainty, and so would entail numerous solutions of our model problem to quantify the uncertainty in our quantity of interest. Finally, as we investigate our model, it might be advantageous to change the assumptions of our cyclone model, or incorporate updated forecasts or indeed play with adding different types of uncertainty to the various input data streams. This is where live programming and real time interaction with our model becomes paramount. 