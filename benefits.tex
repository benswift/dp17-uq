\subsection*{BENEFIT}


Solutions to the core scientific problems identified with the three main aims of our project will be of high impact and will lead to the future development of  software infrastructure for environmental disaster modelling including the quantifying of uncertainty, and for the software engineering use of live-coding in general. As noted above, we anticipate being able to win scholarship funding to add an additional three PhD students to the project team, and we will also supervise several undergraduate projects.  We assess in our budgeting that an ARC investment of about  \$1Million will be accompanied by about  \$590K of cash and in-kind contribution from the ANU. The larger team of APA PhD students and undergraduate project students would increase this in-kind contribution to a nominal level of \$850K. 
  

In a world that is prey to a large number of environmental risks,
the societal benefit of improved computational modelling for disaster management is huge.
This project addresses the mathematical challenges (through new approaches
to uncertainty quantification) and the software engineering challenges (through live coding
 for rapid prototyping) and the systems support challenges (coping with volatile and 
 rapidly changing systems environments) of this very important area.
 
In addition to the academic benefits of the research program described here, there are potential follow-on projects in close
collaboration with industry and possibly even the seeding of spin-off enterprises that develop and deliver guaranteed modelling infrastructure to disaster management
agencies.

\iffalse
This research is aims to unlock the power of sophisticated
computational simulation incorporating uncertainty for
\emph{interactive} use.  Although we concentrate our research on
simulation support for disaster response, the ultimate potential of
this work is to eventually empower domain experts from a broad range
of areas to better use the high-performance computing power which is
now available to them. We envision a future where performing a complex
flood model or disaster simulation is as interactive and \emph{alive}
as flicking through photos on a tablet.\\



There are several benefits arising from the successful completion of
this project. They include, economic, societal as well as
environmental along with the generation of new knowledge:
\begin{itemize}
\item Reduced economic losses from disasters such as flooding
\item Reduced property damage from disasters
\item Reduced loss of life from disasters
\item New knowledge in the understanding of disaster modelling, and
  hence forecasting.
\end{itemize}
\fi