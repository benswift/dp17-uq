\documentclass[a4paper,fontsize=12pt]{scrartcl}
\pagestyle{empty} % suppress page numbers

% set margins
\usepackage[left=1.0cm,top=1.0cm,right=1.0cm,bottom=1.0cm,bindingoffset=0cm]{geometry}
% \usepackage[left=0.5cm,top=0.5cm,right=0.5cm,bottom=0.5cm,bindingoffset=0cm]{geometry}

%%%%%%%%%
% fonts %
%%%%%%%%%

% use the Alegreya font
%\usepackage{Alegreya}
%\usepackage{AlegreyaSans}

% a times new roman clone
% \usepackage{times}

% true Times New Roman with XeLaTeX
%\usepackage{fontspec}
%\setmainfont{Times New Roman}

\usepackage[svgnames,hyperref]{xcolor}
\usepackage{url}
\usepackage{graphicx}

\usepackage{amsmath,amssymb,amsthm}
\usepackage{tikz}
\usetikzlibrary{shapes,arrows}
\usepackage[utf8]{inputenc}

\usepackage[%
backend=biber,
style=authoryear, % numeric-comp
maxbibnames=10,
url=false,
doi=false]{biblatex}

% make bibliography a bit smaller if necessary
\renewcommand*{\bibfont}{\footnotesize} 
% \footnotesize should be 10pt with a standard 12pt class, 
% see https://en.wikibooks.org/wiki/LaTeX/Fonts#Sizing_text

% \AtEveryBibitem{\clearfield{month}}
% \AtEveryCitekey{\clearfield{month}}

\addbibresource{references.bib}

\usepackage[english=british,threshold=15,thresholdtype=words]{csquotes}
\SetCiteCommand{\parencite}

\newenvironment*{smallquote}
{\quote\small}
{\endquote}
\SetBlockEnvironment{smallquote}

\usepackage[%
unicode=true,
hyperindex=true,
bookmarks=true,
colorlinks=true, % change to false for final
pdfborder=0,
allcolors=DarkBlue,
% plainpages=false,
pdfpagelabels,
hyperfootnotes=true]{hyperref}

\usepackage{todonotes}

\author{}

\date{\today}

\begin{document}

\section*{E1. Justification of funding requested from the ARC}
\label{sec:justificatoin}

%==========================
\subsection*{Personnel}


%------------------------------------------------
\subsubsection*{Research Associate – Computer Science (Level B3, 1.0 FTE). Three year cost \$418785.}

This Computer Science Research Associate will need to have overall technical oversight of this project. It needs to be a relatively senior postdoctoral appointment because of the range of interdisciplinary research expertise in software engineering, human computer interaction and mathematics required. In particular, prior expertise in the Extempore programming language and environment would be valued for this position and would warrant the higher level B3 appointment.


%------------------------------------------------
\subsubsection*{Research Associate – Mathematics (Level B2, 1.0 FTE).  Three year cost \$376232.}

This Mathematics Research Associate will need to have specialist expertise in the area of sparse grids and uncertainty quantification at the level of a PhD.  Research experience in the engineering of substantial modelling software will also be expected. Level B1 is appropriate for a beginning post doc. 

%------------------------------------------------
\subsubsection*{PhD/HDR stipend – Computer Science.  Three year cost \$77583.}

The contributions from a PhD student to the Computer Science component of the project is essential to the wider outcomes of this project.
Although the university will support some postgraduates, this is a competitive
process across the whole university, and so is not guaranteed success. Having one guaranteed stipend in Computer Science will enable us to advertise for the best postgrads for the project and will guarantee that we are able to fulfil the outcomes of the project. 

The Computer Science PhD project will be primarily concerned with methodological aspects of live-coding in the software engineering process. The project will start by building mock disaster response game scenarios and then conducting experiments with prototype software and humans playing these game scenarios. As the main software suite comes together these scenarios will become more realistic. We will need an applicant who has experience in live programming, a specialised skill. 

%------------------------------------------------
\subsubsection*{PhD/HDR stipend - Mathematics. Three year cost \$77583.}

The contributions from a PhD student to the Mathematics component of the project is essential to the wider outcomes of this project.
Although the university will support some postgraduates, this is a competitive
process across the whole university, and so is not guaranteed success. Having one guaranteed stipend in Mathematics will enable us to advertise for the best postgrads for the project and will guarantee that we are able to fulfil the outcomes of the project. 

The Mathematics PhD project will be primarily concerned with flood-surge modelling. At present our major expertise is in tsumami modelling with uncertain bathymetry data. This project will focus on down-stream surge modelling and will model forcing functions from disaster swell phenomena. 

%==========================
\subsection*{Travel}

%------------------------------------------------
\subsubsection*{Domestic conference - computer science. Three year cost \$6,778.}

We request funds to cover the cost of the computer science research associate to attend important domestic conferences.  The ANU will cover the cost of one of the CIs, Gardner or Strazdins, to attend the same domestic conferences as the research associate. 

Because of the interdisciplinary computer science nature of this project, we anticipate that the Australasian Computer Science Conference (ACSC) will be one target destination for publication as it solicits contributions in Simulation, Software Engineering, Scientific Computing, Human Computer Interaction, Programming Languages and Distributed Systems. In 2017, CI Strazdins was the Program Chair of ACSC. 
The Australian Conference on Human-Computer Interaction, OzCHI is the premier domestic conference in which to present Human Computer Interaction research. The Australasian Software Engineering Conference, ASWEC, is the premier domestic software engineering conference. 

\begin{description}


\item[Year 2 - 2019] ACSC.
Location is unknown at the time of writing. Perhaps Adelaide.
Return airfare Canberra- Adelaide (Qantas - Flexi) \$898
Accommodation Adelaide, Stamford Plaza \$169 $\times$ 4 =\$676
Conference registration \$546 (CORE Member Registration)
No per diem. Cost \$2,120.

\item[Year 2 - 2019] OzCHI.
Location is unknown at the time of writing. Perhaps Adelaide.
Return airfare Canberra- Adelaide (Qantas - Flexi) \$898
Accommodation Adelaide, Stamford Plaza \$169 $\times$ 4 =\$676
Conference registration \$760 (non-member early reg)
No per diem. Cost \$2,334.

\item[Year 3 -  2020] ASWEC.
Location is unknown at the time of writing. Perhaps Adelaide.
Return airfare Canberra- Adelaide (Qantas - Flexi) \$898
Accommodation Adelaide, Stamford Plaza \$169 $\times$ 4 =\$676
Conference registration \$750 (non-member early reg)
No per diem. Cost \$2,324.


\end{description}



%------------------------------------------------
\subsubsection*{Domestic Conference - mathematics. Three year cost \$7,288.}

We request funds to cover the cost of the mathematics research associate to attend important domestic conferences.  The ANU will cover the cost of one of the CIs, Roberts or Hegland to also attend the same domestic conferences as the research associate. 

The International Congress on Modelling and Simulation, MODSIM is the premier domestic conference in which to present advance in modelling and simulation. 
The Computational Techniques and Applications Conference, CTAC, is the premier domestic computational mathematics conference.

\begin{description}
\item[Year 1 - 2018] Modsim2018. Destination unknown. Perhaps Adelaide.
Paramount modelling and simulation conference in Australia. It provides an excellent venue for the Mathematics Research Associate to present the results of our research. 
Registration fee of \$850, 
Return airfare Canberra- Adelaide (Qantas - Flexi) \$898,
Accommodation Adelaide, Stamford Plaza \$169 $\times$ 4 =\$676
No per diem. Cost \$2,424.



\item[Year 2 - 2018] CTAC2019.
Paramount computational science conference in Australia. It will provide an excellent venue to present the results of our research. Location unknown. Perhaps Adelaide.
Estimated costs are based on 
Registration fee of \$850, 
Return airfare Canberra- Adelaide (Qantas - Flexi) \$898,
Accommodation Adelaide, Stamford Plaza \$169 $\times$ 4 =\$676
No per diem. Cost \$2,424.

\item[Year 3 - 2019] Modsim2020.
Paramount modelling and simulation conference in Australia. It will provide an excellent venue to present the results of our research. Location unknown. Perhaps Adelaide.
Costs are based on 
Registration fee of \$850, 
Return airfare Canberra- Adelaide (Qantas - Flexi) \$898,
Accommodation Adelaide, Stamford Plaza \$169 $\times$ 4 =\$676
No per diem. Cost \$2,440.




\end{description}

%------------------------------------------------
\subsubsection*{International conference - computer science. Three year cost \$16,207.}


We request funds to cover the cost of the computer science research associate to attend important international conferences.    
The ANU will cover the cost of one of the CIs, Gardner or Strazdins to also attend the same international conferences as the research associate. 

As part of the overseas trips to the United States (in year 2) we plan to also visit our collaborators at the Sandia National Laboratories in Livermore. The main international conference targets will be SC (``Supercomputing''), CHI (``Computer Human Interaction'') and ICSE (``International Conference on Software Engineering'').


\begin{description}


\item[Year 2 - 2019] ICSE 2019.
Montreal, Canada
Airfare Canberra - Montreal \$3,372
5 nights accomm. Hotel Le Dauphin \$203 $\times$ 5 = \$1,015
Registration: \$1,028 (IEEE Member rate)
Total Cost \$5,415

\item[Year 2 - 2019] SC 2019
Location: unknown
Perhaps Washington, DC, USA
Return Airfare Canberra - Washington DC, stopover in Dallas  \$2545
5 nights accomm. Washington DC, Hotel Harrington \$1173
Return Airfare Dallas-Albuqerque \$811
Hotel - Albuquerque 5 $\times$  \$150 = \$750,
Per diem 50\% ATO; cost group 5; 50\% AT0= \$115, 5 nights = \$560. Total Cost \$5,839.

\item[Year 3 - 2020] CHI 2020.
Location: unknown 
Perhaps Pittsburg, PA, USA
Registration \$850 
Airfare Canberra - Pittsburg \$3361. 
5 nights accomm. Hilton Garden Inn, Pittsburgh \$805 
Total Cost \$5,016.



\end{description}


%------------------------------------------------
\subsubsection*{International conference - mathematics. Three year cost \$.}


We request funds to cover the cost of the mathematics research associate to attend important international conferences.  The ANU will cover the cost of one of the CIs, Roberts or Hegland to also attend the same international conferences as the research associate. 

As part of the overseas trips to the United States (years 1 and 2) we plan to also visit our collaborators at the Sandia National Laboratories in Albuquerque. 

The SIAM Conference on Computational Science and Engineering, SIAM CSE is the premier international conference in which to present advances in computational algorithms and methods. The SIAM Conference on Uncertainty Quantification, SIAM UQ, is the premier international uncertainty quantification conference.

\begin{description}
\item[Year 1 - 2017] SIAM CSE 2017.
Costs based on registration \$640 (US\$450), 
Travel (to Atlanta, Georgia, USA, with stopover in Albuquerque) \$3280, 
Accommodation Atlanta Hilton, \$1005 (\$201 p. night).  
Hotel - Albuquerque 5 $\times$  \$150 = \$750,
Per Diem: 50\% ATO; cost group 5; 50\% AT0 =\$115, 5 nights =\$560, Cost \$6,235.

\item[Year 2 - 2018] SIAM UQ 2018.
Main international conference for the presentation of research in uncertainty quantification. Location unknown.
Estimated costs based on 
Registration \$600,
Travel (to US, with stopover in Albuquerque) \$3500, 
Conference accommodation \$1000.
Hotel - Albuquerque 5 $\times$  \$150 = \$750,
Per diem 50\% ATO; cost group 5; 50\% AT0= \$115, 5 nights = \$560. Cost \$6,410.

\item[Year 3 - 2019]  SIAM CSE 2019.
Main international conference for the presentation of research in computational science and engineering. Location unknown.
Estimated costs based on 
Registration \$600,
Travel (to US) \$3500, 
Accommodation \$1000.
Per diem 50\% ATO; cost group 5; 50\% AT0= \$115, 5 nights = \$560. Cost \$5,660.

\end{description}

%==========================
\subsection*{Equipment}

%------------------------------------------------
\subsubsection*{Data server. Cost \$2,384.}

We will be collecting a large amount of data from our simulations and from our HCI experiments, which will need to be analysed and stored over extended periods. In particular we will need to analyse video data.  For this we will need a server specifically for the project. A typical machine is the Dell 462-8700 Precision Tower 5810 Workstation features 3.5 GHz Intel Xeon E5-1620 v3 Processor, 16GB of 2133 MHz DDR4 ECC RDIMM RAM, quoted as \$2,622 inclusive of GST.  


%------------------------------------------------
\subsubsection*{Video cameras. Total cost \$1,350.}

We need 3 video cameras to film human experiments, for instance, GOPRO - 8MP - HERO+ LCD – CHDHB-101, quoted as \$450. 

%------------------------------------------------
\subsubsection*{Video editing software. Cost \$1,000.}

We will need at least 2 licences for professional video editing software such as Final Cut Pro. Quote \$500/licence. 





\end{document}

% Local Variables:
% TeX-engine: xetex
% End: