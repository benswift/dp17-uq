\documentclass[a4paper,twoside,12pt,compact]{article}
%\documentclass[a4paper,fontsize=12pt]{scrartcl}
\pagestyle{empty} % suppress page numbers

% set margins
\usepackage[left=1.1cm,top=1.1cm,right=1.1cm,bottom=1.1cm,bindingoffset=0cm]{geometry}
%\usepackage[left=1.0cm,top=1.0cm,right=1.0cm,bottom=1.0cm,bindingoffset=0cm]{geometry}
% \usepackage[left=0.5cm,top=0.5cm,right=0.5cm,bottom=0.5cm,bindingoffset=0cm]{geometry}

%%%%%%%%%
% fonts %
%%%%%%%%%

% use the Alegreya font
%\usepackage{Alegreya}
%\usepackage{AlegreyaSans}

% a times new roman clone
\usepackage{times}

% true Times New Roman with XeLaTeX
%\usepackage{fontspec}
%\setmainfont{Times New Roman}

\usepackage[svgnames,hyperref]{xcolor}
\usepackage{url}
\usepackage{graphicx}

\usepackage{amsmath,amssymb,amsthm}
\usepackage{tikz}
\usetikzlibrary{shapes,arrows}
\usepackage[utf8]{inputenc}

\usepackage[%
backend=biber,
style=authoryear, % numeric-comp
maxbibnames=10,
url=false,
doi=false]{biblatex}

% make bibliography a bit smaller if necessary
\renewcommand*{\bibfont}{\footnotesize} 
% \footnotesize should be 10pt with a standard 12pt class, 
% see https://en.wikibooks.org/wiki/LaTeX/Fonts#Sizing_text

% \AtEveryBibitem{\clearfield{month}}
% \AtEveryCitekey{\clearfield{month}}

\addbibresource{references.bib}

\usepackage[english=british,threshold=15,thresholdtype=words]{csquotes}
\SetCiteCommand{\parencite}

\newenvironment*{smallquote}
{\quote\small}
{\endquote}
\SetBlockEnvironment{smallquote}

\usepackage[%
unicode=true,
hyperindex=true,
bookmarks=true,
colorlinks=true, % change to false for final
pdfborder=0,
allcolors=DarkBlue,
% plainpages=false,
pdfpagelabels,
hyperfootnotes=true]{hyperref}

\usepackage{todonotes}

\author{}

\date{\today}

\begin{document}

\section*{E1. Justification of funding requested from the ARC}
\label{sec:justificatoin}

%==========================
\subsection*{Personnel}


%------------------------------------------------
\subsubsection*{Research Associate – Computer Science (Level B3, 1.0 FTE). Three year cost \$418,785.}

This Computer Science Research Associate will need to have overall technical oversight of this project. It needs to be a relatively senior postdoctoral appointment because of the range of interdisciplinary research expertise in software engineering, human computer interaction and mathematics required. In particular, prior expertise in the Extempore programming language and environment would be valued for this position and would warrant the higher level B3 appointment.


%------------------------------------------------
\subsubsection*{Research Associate – Mathematics (Level B1, 1.0 FTE).  Three year cost \$376,233.}

This Mathematics Research Associate will need to have specialist expertise in the area of sparse grids and uncertainty quantification at the level of a PhD.  Research experience in the engineering of substantial modelling software will also be expected. Level B1 is appropriate for a beginning post doc. 

%------------------------------------------------
\subsubsection*{PhD/HDR stipend – Computer Science;  3.5 year cost \$92,050.}

The contributions from a PhD student to the Computer Science component of the project is essential to the wider outcomes of this project.
Although the university will support some postgraduates, this is a competitive
process across the whole university, and so is not guaranteed success. Having one guaranteed stipend in Computer Science will enable us to advertise for the best postgrads for the project and will guarantee that we are able to fulfil the outcomes of the project. 

The Computer Science PhD project will be primarily concerned with methodological aspects of live-coding in the software engineering process. The project will start by building mock disaster response game scenarios and then conducting experiments with prototype software and humans playing these game scenarios. As the main software suite comes together these scenarios will become more realistic. We will seek an applicant who has experience in live programming, a specialised skill. 

Note that the common completion time for PhD candidates in computer science is just over 3.5 years and the extension of the stipend for 6 months beyond 3 years needs to be budgeted for.  

%------------------------------------------------
\subsubsection*{PhD/HDR stipend - Mathematics; 3.5 year cost \$92,050.}

The contributions from a PhD student to the Mathematics component of the project is essential to the wider outcomes of this project.
Although the university will support some postgraduates, this is a competitive
process across the whole university, and so is not guaranteed success. Having one guaranteed stipend in Mathematics will enable us to advertise for the best postgrads for the project and will guarantee that we are able to fulfil the outcomes of the project. 

This Mathematics PhD project will be primarily concerned with flood-surge modelling. It will focus on down-stream surge modelling and will model forcing functions from disaster swell phenomena with quantified uncertainty. 

As with the computer science PhD stipend, the extension of the stipend for 6 months beyond 3 years must be budgeted for.

%==========================
\subsection*{Travel}

%------------------------------------------------
\subsubsection*{Domestic conference - computer science. Three year cost \$5,258.}

We request funds to cover the cost of the computer science research associate to attend important domestic conferences.  The ANU will cover the cost of one of the CIs, Gardner or Strazdins, to attend the same domestic conferences as the research associate. 

Because of the interdisciplinary computer science nature of this project, we anticipate that the Australasian Computer Science Week conferences (ACSW) will be one target destination for publication as it has tracks soliciting contributions in Simulation, Software Engineering, Scientific Computing, Human Computer Interaction, Programming Languages and Distributed Systems. In 2016 and 2017, CI Strazdins was the Program Chair for the main track (ACSC) of ACSW. 
The Australian Conference on Human-Computer Interaction, OzCHI is the premier domestic conference in which to present Human Computer Interaction research. 

\begin{description}


\item[Year 2 - 2019] ACSW.
Location is unknown at the time of writing. Perhaps Adelaide.
Return airfare Canberra- Adelaide (Qantas - Flexi) \$898
Accommodation Adelaide, Stamford Plaza \$169 $\times$ 4 =\$676
Conference registration \$546 (CORE Member Registration)
Meals and incidentals \$63 $\times$ 4 =\$252, Local transport \$150, 
Total Cost \$2,522.

\item[Year 2 - 2019] OzCHI.
Location is unknown at the time of writing. Perhaps Adelaide.
Return airfare Canberra- Adelaide (Qantas - Flexi) \$898
Accommodation Adelaide, Stamford Plaza \$169 $\times$ 4 =\$676
Conference registration \$760 (non-member early reg)
Meals and incidentals \$63 $\times$ 4 =\$252, Local transport \$150, 
Total Cost \$2,736.




\end{description}



%------------------------------------------------
\subsubsection*{Domestic Conference - mathematics. Three year cost \$5,608.}

We request funds to cover the cost of the mathematics research associate to attend important domestic conferences.  The ANU will cover the cost of one of the CIs, Roberts or Hegland to also attend the same domestic conferences as the research associate. 

The International Congress on Modelling and Simulation, MODSIM is the premier domestic conference in which to present advance in modelling and simulation. 
The Computational Techniques and Applications Conference, CTAC, is the premier domestic computational mathematics conference.

\begin{description}




\item[Year 2 - 2019] CTAC2019.
Paramount computational science conference in Australia. It will provide an excellent venue to present the results of our research. Location unknown. Perhaps Adelaide.
Estimated costs are based on 
Registration fee of \$850, 
Return airfare Canberra- Adelaide (Qantas - Flexi) \$898,
Accommodation Adelaide, Stamford Plaza \$169 $\times$ 4 =\$676
Meals and incidentals \$63 $\times$ 4 =\$252, Local transport \$150, 
Total Cost \$2,826.




\item[Year 3 - 2020] Modsim2020.
Paramount modelling and simulation conference in Australia. It will provide an excellent venue to present the results of our research. Location unknown. Perhaps Brisbane.
Costs are based on 
Registration fee of \$850, 
Return airfare Canberra- Brisbane (Qantas - Flexi) \$874,
Accommodation Brisbane, Royal Hotel on the Park \$164 $\times$ 4 =\$656
Meals and incidentals \$63 $\times$ 4 =\$252, Local transport \$150, 
Total Cost \$2,782.




\end{description}

%------------------------------------------------
\subsubsection*{International conference - computer science. Three year cost \$17,510.}


We request funds to cover the cost of the computer science research associate to attend important international conferences.    
The ANU will cover the cost of one of the CIs, Gardner or Strazdins to also attend the same international conferences as the research associate. 

The major international conference for High Performance Computing us SC (``Supercomputing''), the major international conference for Human Computer Interaction is CHI, the major international conference with interests in live software engineering is possible ICSE 5 (``International Conference on Software Engineering''). We have presented at all of these venues (including at the inaugural workshop on Live Programming at ICSE) and would seek to have visibility of our research outcomes there. 

As part of the overseas trips to the United States (in year 2) we plan to also visit our collaborators at the Sandia National Laboratories in Livermore. 




\begin{description}




\item[Year 2 - 2019] SC 2019 and Sandia Labs.
Location: unknown
Perhaps Washington, DC, USA
Return Airfare Canberra - Washington DC, stopover in Dallas  \$2545
5 nights accomm. Washington DC, Hotel Harrington \$1173
Return Airfare Dallas-Albuqerque \$811
5 nights Hotel - Albuquerque 5 $\times$  \$150 = \$750,
Per diem 50\% ATO; cost group 5; 50\% AT0= \$120, 10 nights = \$1200. 
Total Cost \$6,479.

\item[Year 3 - 2020] CHI 2020.
Location: unknown 
Perhaps Pittsburg, PA, USA
Registration \$850 
Airfare Canberra - Pittsburg \$3361. 
5 nights accomm. Hilton Garden Inn, Pittsburgh \$805 
Total Cost \$5,016.

\item[Year 4 - 2021] ICSE 2021.
Montreal, Canada
Airfare Canberra - Montreal \$3,372
5 nights accomm. Hotel Le Dauphin \$203 $\times$ 5 = \$1,015
Registration: \$1,028 (IEEE Member rate),
Per Diem: 50\% ATO; cost group 5; 50\% AT0 =\$120, 5 nights =\$600,
Total Cost \$6,015


\end{description}


%------------------------------------------------
\subsubsection*{International conference - mathematics. Three year cost \$20,600.}


We request funds to cover the cost of the mathematics research associate to attend important international conferences.  The ANU will cover the cost of one of the CIs, Roberts or Hegland to also attend the same international conferences as the research associate. 

As part of the overseas trips to the United States (years 1 and 3) we plan to also visit our collaborators at the Sandia National Laboratories in Albuquerque. 

The SIAM Conference on Computational Science and Engineering, SIAM CSE is the premier international conference in which to present advances in computational algorithms and methods. The SIAM Conference on Uncertainty Quantification, SIAM UQ, is the premier international uncertainty quantification conference.

\begin{description}
\item[Year 1 - 2018] SIAM CSE 2018 and Sandia Labs.
Registration \$640, 
Destination unknown, assume Pittsburg.
Airfare Canberra - Pittsburg \$3361,
Return Airfare Dallas-Albuqerque \$811,
5 nights accomm. Hilton Garden Inn, Pittsburgh \$805,  
5 nights Hotel - Albuquerque 5 $\times$  \$150 = \$750,
Per Diem: 50\% ATO; cost group 5; 50\% AT0 =\$120, 10 nights =\$1,200, 
Total Cost \$7,597.

\item[Year 2 - 2019] SIAM UQ 2019.
Main international conference for the presentation of research in uncertainty quantification. Location unknown, assume Pittsburg.
Registration \$640,
Destination unknown, assume Pittsburg.
Airfare Canberra - Pittsburg \$3361,
5 nights accomm. Hilton Garden Inn, Pittsburgh \$805,  
Per diem 50\% ATO; cost group 5; 50\% AT0= \$120, 5 nights = \$600. 
Total Cost \$5,406.

\item[Year 3 - 2020]  SIAM CSE 2020 and Sandia Labs.
Main international conference for the presentation of research in computational science and engineering. Location unknown, assume Pittsburg.
Registration \$640,
Destination unknown, assume Pittsburg.
Airfare Canberra - Pittsburg \$3361,
Return Airfare Dallas-Albuqerque \$811,
5 nights accomm. Hilton Garden Inn, Pittsburgh \$805,  
5 nights Hotel - Albuquerque 5 $\times$  \$150 = \$750,
Per diem 50\% ATO; cost group 5; 50\% AT0= \$120, 10 nights = \$1200. 
Total Cost \$7,597.

\end{description}

%==========================
\subsection*{Equipment}

%------------------------------------------------
\subsubsection*{Data server. Cost \$3,607}

We will be collecting a large amount of data from our simulations and from our HCI experiments, which will need to be analysed and stored over extended periods. In particular we will need to analyse video data.  For this we will need a server specifically for the project. A typical machine is theDell Precision Rack 7000 Workstation 2.43 GHz Intel® Xeon® E5-2620 Processor (6 cores), I6GB 2400MHz DDR4 RDIMM RAM 500GB  Hard Drive
\$2,779 US =  \$3,607 AUSD.



%------------------------------------------------
\subsubsection*{Video cameras. Total cost \$1,350.}

We need 3 video cameras to film human experiments, for instance, GOPRO - 8MP - HERO+ LCD – CHDHB-101, quoted as \$450. 

%------------------------------------------------
\subsubsection*{Video editing software. Cost \$1,000.}

We will need at least 2 licences for professional video editing software such as Final Cut Pro. Quote \$500/licence. 

%------------------------------------------------




\end{document}

% Local Variables:
% TeX-engine: xetex
% End:
