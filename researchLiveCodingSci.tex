\subsubsection*{Live Coding of Scientific Simulation}

Live coding is a term which has been used to denote systems which
support the direct intervention of the programmer in a program's
run-time state. It can be thought of as an extreme version of the
agile programming methodology~\parencite{fowlerAgile2001}, where
code changes are hot-swapped into running programs, allowing for
extremely fast exploration and iteration of new ideas and system
updates. As the ambition of live-coding has grown, support systems and
languages have evolved to, for example, create, modify and interact
with music and hardware devices in real time. Such an approach has
been termed `\emph{with-time}
programming'~\parencite{sorensen2010programming}. This project will
make use of the \emph{Extempore} software
environment\footnote{\url{http://extempore.moso.com.au}} which has
been used for live modification and real-time visualisation of
particle-in-cell (PIC) plasma physics simulation codes, with
negligible performance overhead compared to batch-mode execution in
C~\parencite{swiftLive2016}. This allows the scientist to modify the
domain size/shape, the initial and boundary conditions, and various
other parameters while the simulation is running, with live visual
feedback.

The Extempore software environment is a key tool for this project as
it allows us to fine tune our suite of simulation software for the
specific requirements of the task domain.


