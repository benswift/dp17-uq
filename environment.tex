
\subsection*{FEASIBILITY}



\noindent{\bf Project design:}
This research has been framed around problems that address the three main aims of the project. The core scientific objectives of each aim will be able to be achieved with reduced models, making them clearly realisable within the 3.5 year timeframe.
The expertise of the CIs is unique and well aligned with this project. All of the CIs know each other well and all have sufficient research capacity to deliver their fractional engagement with the project.

Apart from the CI's, there will be two postdoctoral research associates and five PhD students (two funded from this project and three funded from competitive scholarships): 
one postdoctoral fellow and two PhD students in 
Mathematics will study the simulation (inundation) equations in realistic topographies together with the calculation of uncertainty (with one student focussing primarily on the simulations and the other primarily on the associated quantification of uncertainty). 
In Computer Science, one postdoctoral fellow and three PhD students will study the application of live coding to uncertainty quantification of the inundation simulations, with one PhD student focusing on human-in-the-loop optimisation with expert participants, one PhD student focussing on HPC systems optimisation, 
and the third PhD student studying the wider human-computer-interaction context of real-time disaster management.
Expert participants in our studies will be solicited from local emergency authorities (ACT Emergency
Services, the Australian Maritime and Safety Authority and the Australian Federal Police) and from local environmental scientists (ANU Fenner School, ANU Research School of Earth Sciences, Geosciences Australia and the CSIRO).

The anticipated duration of PhD study in computer science and mathematics is slightly over 3.5 years and this grant application has planned and budgeted for 3.5 years for the two ARC-funded PhD scholarships. In the fourth year of the project, the two research associates will have finished their terms, but the CIs will continue to supervise all of the PhD students through to completion.  

We anticipate a number of Honours and other student projects concerned with the overall context of our research as there are many 
fascinating mathematical and computing problems involved in developing robust simulations of environmental disasters.\\

\noindent{\bf Environment:}
The Australian National University is a research-intensive university
of high international standing where the quality of research in the ERA categories of 
 Information and Computing
Sciences (08) and Mathematical Sciences (01) have been assessed as
being at the highest level of 5. At ANU, the Research School of
Computer Science and the Mathematical Sciences Institute have a long
and deep collaboration in numerical and applied mathematics and a new building is presently being constructed at ANU to
locate these two schools together. This co-location of academics in these two
areas will be particularly fruitful for the present project.\\


\noindent{\bf Facilities:}
The ANU is well-endowed with suitable facilities to carry out this project. Of particular note is the new building mentioned above and the HPC facilities, and associated support, located on campus.
Since the early 1980s, ANU has housed the
largest supercomputers in Australia. The cloud computing facilities at the NCI national
facility, located at ANU, will provide one of the testbeds for this
project.




