% - Outline the adequacy and opportunities within the Research Environment in your relevant department, school or research group, and the extent to which it will provide for knowledge growth, innovation, collaboration, mentoring and student training
% - Describe the existing, or developing, research environment within the Administering Organisation and collaborating Organisation(s) which will enable this Project
% - Describe how the Project aligns with the Administering Organisation’s research plans and strategies.
\iffalse
\subsection*{RESEARCH ENVIRONMENT}
\fi

\subsection*{FEASIBILITY}

\iffalse The project is feasible due to the rigorous design with distinct tasks
and timelines which address the identified, well understood challenges
in this discipline. 
\fi

\iffalse
The project brings together a multi-disciplinary
team of experts in mathematics, software engineering, human computer interaction and high
performance computing.
\fi

\noindent{\bf Project design:}
As described above, this research has been framed around problems that address the three main aims of the project. The core scientific objectives of each aim will be able to be achieved with reduced models, making them clearly realisable within the three year timeframe.
The expertise of the CIs is unique and well aligned with this project. All of the CIs know each other well and all have sufficient research capacity to deliver their fractional engagement with the project.

Apart from the CI's, postdoctoral staff and students will be aligned as follows: One postdoctoral fellow and two PhD students in 
Mathematics will study storm surge equations in realistic topographies together with the calculation of uncertainty. 
In Computer Science, one postdoctoral fellow 
and one PhD student will study the application of live coding to uncertainty quantification of the storm surge simulations,
one PhD student will study HPC aspects of the resulting system
 and the remaining PhD student will study the human-computer-interaction context of real-time disaster management. 
Here we will solicit qualitative feedback from expert participants from
emergency authorities in our local area including ACT Emergency
Services, Geosciences Australia, the Australian Maritime and Safety
Authority and the Australian Federal Police.

%Perspectives offered by
%this participant pool will be important in extrapolating our study
%results to real world disaster-management.

We anticipate a number of Honours and other student projects concerned with the overall context of our research as there are many 
fascinating mathematical and computing problems involved in developing robust simulations of envinromental disasters.\\

\noindent{\bf Environment:}
The Australian National University is a research-intensive university
of high international standing where the quality of research in the ERA categories of 
 Information and Computing
Sciences (08) and Mathematical Sciences (01) have been assessed as
being at the highest level of 5. At ANU, the Research School of
Computer Science and the Mathematical Sciences Institute have a long
and deep collaboration in numerical and applied mathematics and new building is presently being constructed at ANU to
locate these two schools together. This co-location of academics in these two
areas will be particularly fruitful for the present project.\\


\noindent{\bf Facilities:}
The ANU is well-endowed with suitable facilities to carry out this project. Of particular note is the new building mentioned above and the HPC facilites, and associated support, located on campus.
Since the early 1980s, ANU has housed the
largest supercomputers in Australia. The cloud computing facilities at the NCI National
facilities, located at ANU, will provide one of the testbeds for this
project.




