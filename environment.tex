% - Outline the adequacy and opportunities within the Research Environment in your relevant department, school or research group, and the extent to which it will provide for knowledge growth, innovation, collaboration, mentoring and student training
% - Describe the existing, or developing, research environment within the Administering Organisation and collaborating Organisation(s) which will enable this Project
% - Describe how the Project aligns with the Administering Organisation’s research plans and strategies.
\iffalse
\subsection*{RESEARCH ENVIRONMENT}
\fi

\subsection*{FEASIBILITY}

\iffalse The project is feasible due to the rigorous design with distinct tasks
and timelines which address the identified, well understood challenges
in this discipline. 
\fi

\iffalse
The project brings together a multi-disciplinary
team of experts in mathematics, software engineering, human computer interaction and high
performance computing.
\fi

This research will be framed around problems that address the three main aims of the project. Solutions to these problems will be of high impact and will be core to the development of an entire software infrastructure for disaster modelling as shown in Figure 1.


The Australian National University is a research-intensive university
of high international standing where the quality of research in the ERA categories of 
 Information and Computing
Sciences (08) and Mathematical Sciences (01) have been assessed as
being at the highest level of 5. At ANU, the Research School of
Computer Science and the Mathematical Sciences Institute have a long
and deep collaboration in numerical and applied mathematics, notably  linked to the supercomputing
facilities on campus. Indeed, since the early 1980s, ANU has housed the
largest supercomputers in Australia and the cloud computing facilities at the NCI National
facilities, located at ANU, will provide one of the testbeds for this
project.

A new building is presently being constructed at ANU to
locate the Mathematical Sciences Institute with the Research
School of Computer Science. This co-location of academics in these two
areas will be particularly fruitful for the present project.
