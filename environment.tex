\subsection*{RESEARCH ENVIRONMENT}
% - Outline the adequacy and opportunities within the Research Environment in your relevant department, school or research group, and the extent to which it will provide for knowledge growth, innovation, collaboration, mentoring and student training
% - Describe the existing, or developing, research environment within the Administering Organisation and collaborating Organisation(s) which will enable this Project
% - Describe how the Project aligns with the Administering Organisation’s research plans and strategies.

The Australian National University is a research-intensive university
of high international standing. The quality of the research at ANU has
been reflected in every one of the ERA evaluation exercises where,
relevant to the current proposal, both Information and Computing
Sciences (08) and Mathematical Sciences (01) have been assessed as
being at the highest level of 5. At ANU, the Research School of
Computer Science and the Mathematical Sciences Institute have a long
and deep collaboration in numerical and applied mathematics, notably
in a number of ''Area'' projects linked to the supercomputing
facilities on campus. Indeed, since the early 1980s ANU has housed the
largest supercomputers in Australia and the present National
Computational Infrastructure is located on campus.

Over the next 2 years, a new building will be constructed at ANU to
locate the Mathematical Sciences Institute with part of the Research
School of Computer Science. This co-location of academics in these two
areas will be particularly fruitful for the present project. Indeed,
it is possible that this project will become a showcase of cooperation
between these two Schools and that some of the more visible and
interactive components of the project will be strongly featured in
outreach activities.

Furthermore, the cloud computing facilities at the NCI National
facilities will provide an ideal testbed for the
project. Virtualization is necessary for tuning Extempore, and the
size of the clusters will permit large-scale simulations by project
staff.
