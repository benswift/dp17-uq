\subsection*{COMMUNICATION OF RESULTS}
% - Outline plans for communicating the research results to other researchers and the broader community, including but not limited to scholarly and public communication and dissemination.

The ANU expects publications at the highest levels of international
journals and conferences. We will communicate the results of this
project by publishing in top venues such as the ``SIAM Journal of
Scientific Computing'',``Parallel Computing'', and the ``Journal of
Computational Science''. In computer science, refereed publications
associated with the top conferences are more prestigious than
journals. Our publication targets will be the ACM Conference on Human
Factors in Computing Systems (CHI), OOPSLA, ICSE, VL/HCC,
Supercomputing (SC), IEEE International Parallel and Distributed
Processing Symposium (IPDPS), Computer Supported Cooperative Work
(CSCW) and IEEE Information Visualisation. In both disciplines, we
also value the community and high quality of Australian conferences,
and we will be submitting work to OzCHI, ASWEC, Computational
Techniques and Applications Conference (CTAC) and the International 
Congress on Modelling and Simulation (MODSIM). We will present 
our work at the International SIAM conferences in Computational Science and Engineering and Uncertainty Quantification. 


In addition, the source code contributions of this project will be
released to the public. Both the Extempore live programming system
(\url{https://github.com/digego/extempore}) and the
 AnuGA~\parencite{anugamanual,nielsen2005hydrodynamic} 
shallow
water simulation package
(\url{https://github.com/GeoscienceAustralia/anuga_core}) are
available on GitHub under MIT and GPLv2 licences respectively.
Parallel Sparse Grid Combination codes (ParSGCT) are available from CI
Strazdins' website.  We are committed to accessible and reproducible
computational science, and support these goals by using free software
licences and developing our code in the open on GitHub.

