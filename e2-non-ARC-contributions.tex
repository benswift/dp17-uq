\documentclass[a4paper,fontsize=12pt]{scrartcl}
\pagestyle{empty} % suppress page numbers

% set margins
\usepackage[left=1.0cm,top=1.0cm,right=1.0cm,bottom=1.0cm,bindingoffset=0cm]{geometry}
% \usepackage[left=0.5cm,top=0.5cm,right=0.5cm,bottom=0.5cm,bindingoffset=0cm]{geometry}

%%%%%%%%%
% fonts %
%%%%%%%%%

% use the Alegreya font
\usepackage{Alegreya}
\usepackage{AlegreyaSans}

% a times new roman clone
% \usepackage{times}

% true Times New Roman with XeLaTeX
%\usepackage{fontspec}
%\setmainfont{Times New Roman}

\usepackage[svgnames,hyperref]{xcolor}
\usepackage{url}
\usepackage{graphicx}

\usepackage{amsmath,amssymb,amsthm}
\usepackage{tikz}
\usetikzlibrary{shapes,arrows}
\usepackage[utf8]{inputenc}

\usepackage[%
backend=biber,
style=authoryear, % numeric-comp
maxbibnames=10,
url=false,
doi=false]{biblatex}

% make bibliography a bit smaller if necessary
\renewcommand*{\bibfont}{\footnotesize} 
% \footnotesize should be 10pt with a standard 12pt class, 
% see https://en.wikibooks.org/wiki/LaTeX/Fonts#Sizing_text

% \AtEveryBibitem{\clearfield{month}}
% \AtEveryCitekey{\clearfield{month}}

\addbibresource{references.bib}

\usepackage[english=british,threshold=15,thresholdtype=words]{csquotes}
\SetCiteCommand{\parencite}

\newenvironment*{smallquote}
{\quote\small}
{\endquote}
\SetBlockEnvironment{smallquote}

\usepackage[%
unicode=true,
hyperindex=true,
bookmarks=true,
colorlinks=true, % change to false for final
pdfborder=0,
allcolors=DarkBlue,
% plainpages=false,
pdfpagelabels,
hyperfootnotes=true]{hyperref}

\usepackage{todonotes}

\author{}

\date{\today}

\begin{document}

\section*{E2. Details of non-ARC contributions}
\label{sec:justificatoin}

%==========================
\subsection*{Personnel}

%------------------------------------------------
\subsubsection*{A/Prof Henry Gardner (in-kind Salary Level D2, 0.2 FTE)}
A/Professor  Henry Gardner’s salary (20\% of Academic Level D2 at ANU) 
and on-costs of 30\% and cash contribution of 1.55\% 
(difference between ANU and ARC on-cost percentage) are
directly contributed by the Australian National University. 

%------------------------------------------------
\subsubsection*{A/Prof Stephen Roberts (in-kind Salary Level D3, 0.2 FTE)}
A/Professor  Stephen Roberts’ salary (20\% of Academic Level D3 at ANU) 
and on-costs of 30\% and cash contribution of 1.55\% 
(difference between ANU and ARC on-cost percentage) are
directly contributed by the Australian National University. 

%------------------------------------------------
\subsubsection*{A/Prof Peter Strazdins (in-kind Salary Level D2, 0.2 FTE)}
A/Professor  Peter Strazdins’ salary (20\% of Academic Level D2 at ANU) 
and on-costs of 30\% and cash contribution of 1.55\% 
(difference between ANU and ARC on-cost percentage) are
directly contributed by the Australian National University. 

%------------------------------------------------
\subsubsection*{Prof Trygve Hegland (in kind Salary Level E1. 0.2 FTE)}
Professor  Trygve Hegland's salary (20\% of Academic Level E1 at ANU) 
and on-costs of 30\% and cash contribution of 1.55\% 
(difference between ANU and ARC on-cost percentage) are
directly contributed by the Australian National University. 

%%------------------------------------------------
%\subsubsection*{Dr Bert Debusschere (in kind Salary 0.05 FTE)}
%Dr Bert Debusschere's salary (5\% of Technical Staff Salary) 
%and on-costs of 30\% 
%directly contributed by the Sandia National Laboratories. 
%
%
%%------------------------------------------------
%\subsubsection*{Dr John Jakeman (in kind Salary 0.05 FTE)}
%Dr John Jakeman's salary (5\% of Technical Staff Salary) 
%and on-costs of 30\% 
%directly contributed by the Sandia National Laboratories. 




%------------------------------------------------
\subsubsection*{Research Associate – Computer Science (Level B3, 1.0 FTE)}

The ANU will cover the difference between ARC and ANU salary rates.


%------------------------------------------------
\subsubsection*{Research Associate – Mathematics (Level B2, 1.0 FTE)}

The ANU will cover the difference between ARC and ANU salary rates.

%%------------------------------------------------
\subsubsection*{PhD/HDR stipends – Computer Science and Mathematics}

Two APA Scholarship items have been requested from the ARC in E1.
As mentioned in C1, three further APA scholarships will be 
sought by the project through ANU. Two will be for students
hosted by Computer Science and one by Mathematics.


%%------------------------------------------------
%\subsubsection*{PhD/HDR stipend - Mathematics}


%==========================
\subsection*{Travel}

%------------------------------------------------
\subsubsection*{Domestic conference - computer science}

The ANU will cover the cost of one of the CIs, Gardner or Strazdins 
to attend the  same domestic conferences as the computer science 
research associate. 

%The Australian Conference on Human-Computer Interaction, OzCHI 
%is the premier domestic conference in which to present 
%Human Computer Interaction research. 
%The Australasian Software Engineering Conference, ASWEC, 
%is the premier domestic software engineering conference.

\begin{description}
\item[Year 2 - 2019] ACSW. Cost \$2,120  (see E1 justification).

\item[Year 2 - 2019] OzCHI. (see E1) \$2,334 (see E1 justification).

\item[Year 3 -  2020] ASWEC.  Cost \$2324 (see E1 justification).

\end{description}

%------------------------------------------------
\subsubsection*{Domestic Conference - mathematics}

The ANU will cover the cost of one of the CIs, Roberts or Hegland 
to attend the  same domestic conferences as the mathematics 
research associate. 

%The International Congress on Modelling and Simulation, 
%MODSIM is the premier domestic conference in which to present 
%advances in modelling and simulation. 
%The Computational Techniques and Applications Conference, CTAC, 
%is the premier domestic computational mathematics conference.

\begin{description}
\item[Year 1 - 2018] Modsim2018. Cost \$2,424  (see E1 justification).

\item[Year 2 - 2019] CTAC2019. Cost \$2,424  (see E1 justification).

\item[Year 3 - 2020] Modsim2020.  Cost \$2,440  (see E1 justification).


\end{description}

%------------------------------------------------
\subsubsection*{International conference - computer science}


The ANU will cover the cost of one of the CIs, Gardner or Strazdins 
to also attend the same international conferences and Sandia labs as the 
research associate. 

%The  Association for Computing Machinery’s (ACM) SIGPLAN 
%conference on Systems, Programming, Languages and Applications: 
%Software for Humanity (SPLASH) is the premier international 
%conference at the intersection of programming, languages, 
%and software engineering.
%The International Conference on Software Engineering, ICSE, 
%is the premier international software engineering conference.
%The Association for Computing Machinery’s (ACM) CHI conference 
%is the world's premiere conference on Human Factors in 
%Computing Systems.

\begin{description}
\item[Year 2 - 2019] ICSE 2019. Cost \$5,415  (see E1 justification).

\item[Year 2 - 2019] SC 2019.  Cost \$5,839  (see E1 justification).

\item[Year 3 - 2020] CHI 2020.  Cost \$5,016 (see E1 justification).

\end{description}


%------------------------------------------------
\subsubsection*{International conference - mathematics}

The ANU will cover the cost of one of the CIs, Roberts or Hegland 
to also attend the same international conferences and Sandia labs as the 
mathematics research associate. 

%The SIAM Conference on Computational Science and Engineering, 
%SIAM CSE is the premier international conference in which 
%to present advances in computational algorithms and methods. 
%The SIAM Conference on Uncertainty Quantification, SIAM UQ, 
%is the premier international uncertainty quantification conference.

\begin{description}
\item[Year 1 - 2018] SIAM CSE 2018.  Cost \$6,252   (see E1 justification).

\item[Year 2 - 2019] SIAM UQ 2019.  Cost \$6,410  (see E1 justification).

\item[Year 3 - 2020]  SIAM CSE 2020.  Cost \$5,660   (see E1 justification).

\end{description}

%==========================
\subsection*{Equipment}

%------------------------------------------------
%\subsubsection*{Data server}


%------------------------------------------------
%\subsubsection*{Video cameras}


%------------------------------------------------
%\subsubsection*{Video editing software}


%------------------------------------------------
\subsubsection*{Access to NCI Tenjin HPC Cluster. 200k service units}

In this project we will leverage on-demand compute resources, 
such as the National Compute Infrastructure NCI Cloud Tenjin. 
Using these cloud services will further improve the project's 
ability to deliver timely results in high-pressure and 
time-critical decision making scenarios.

A 24 hours high fidelity simulation of a 50-100 km$^2$ catchment 
represented by a 2,500,000 triangular mesh takes 
some 400 hrs  CPU time. When run on the NCI Raijin machine 
with 640 processors, this reduces to 1 hr wall time.
To produce a reasonable surrogate model we will need to run multiple
such jobs off line.  
In the development stage we will apply for enough CPU time
to run up to 500 high fidelity runs on small catchments. 
This equates to 200k CPU hours.   
As such, in each of the project we will apply for at least
200k service units from the ANU allocation of NCI facility.  

Costing based on NCI partner rate of 0.75  of the standard 
NCI rate of 4c/service Unit. Total \$6,000/year. 




\end{document}

% Local Variables:
% TeX-engine: xetex
% End:
