\subsection*{AIMS AND BACKGROUND}

Emergency services responding to environmental disasters (such as
floods and bushfires) increasingly utilise computational simulations
to model and influence critical decisions involving the deployment of
resources. Such simulations involve elaborate mathematical models
overlayed on realistic terrain at scales relevant to local
communities. They need to incorporate field data (weather conditions,
flood levels etc) and they are very demanding computationally,
requiring high-performance computing in order to run effectively. In
order to be useful for decision makers, simulations need to
accommodate rapidly changing scenarios and field data, and they need
to be able to properly quantify and communicate the uncertainty in
their predictions. Developing such simulations requires insights from
applied mathematics, software engineering, high-performance computing
and human-computer interaction.

\iffalse
When systems are disrupted during environmental disasters (for example,
during floods, storm surges or tsunamis) information from
computational simulation is needed urgently to aid decision-making by
emergency services. This is not simply a matter of compute power but
requires insight from applied mathematics, software
engineering, human computer interaction, high performance computing
and data visualisation. All of these facets need to be tightly coupled
and {\bf tested in realistic  decision-making
scenarios} where decision makers collaborate with modelling experts. In
particular, a crucial aspect of advice provided from modellers to
decision makers is to {\bf properly quantify and communicate the
{uncertainty}} in predictions obtained by computer simulations.
This advice needs to be provided in a timely manner and in a context
where the computer support may vary dramatically due to
system outages.
\fi

\subsubsection*{Aims}

This research addresses the need for real-time simulation of
environmental disasters through a {\em live coding} (or ‘live programming’)
approach to {\bf rapid scenario building including uncertainty}. For the
purposes of this project, we will specialise our concerns to
environmental models of inundation with the specific objectives of
\begin{enumerate}

\item {\bf Developing} new inundation models (models of riverine floods,
  storm surges and tsunamis) which use sparse grids and reduced basis
  approaches to quantify uncertainty and dramatically increase the
  speed and usefulness of model predictions.

\item {\bf Harnessing and prototyping} these models using live coding
  infrastructure to accelerate software development and to provide an
  interactive interface to deployed software in a rapidly changing
  environment.

\item {\bf Deploying and benchmarking} these models on realistic,
  heterogeneous, high-performance computational infrastructure,
  including local compute clusters and the cloud.

\end{enumerate}

\iffalse
We will address this grand challenge through a \emph{live coding} (or
`live programming') approach to the development and deployment of
software systems for environmental simulation in the presence of
uncertainty. We will \textbf{develop} new `inundation models' (models
of riverine floods, storm surges and tsunamis) which use sparse grids
and uncertainty quantification to dramatically increase the speed and
usefulness of model predictions and their associated uncertainty. We
will \textbf{deploy} `live' distributed computing infrastructure to
\emph{rapidly prototype} the human interface to these new
environmental simulations and to transform the traditional,
batch-oriented workflow of environmental simulation into a highly
interactive one.
%Indeed live programming offers an unprecedented opportunity
%for interactive parallel program performance evaluation and tuning.
We will \textbf{evaluate team coordination and
decision-making} in the presence of uncertainty through role-playing
scenarios of representative flood and tsunami situations. 
%In these mock
%field trials, we will record (both through instrumented system logs
%and interviews with participants) the interaction patterns and
%protocols of live programmed optimisation of the software modelling
%systems. From a human-computer interaction (HCI) analysis of these
%protocols, we will \textbf{identify} patterns in the interaction
%states and transitions between those states which we will use to
%iterate our software-system development.
\fi


The outcome of the research will enable decision makers to examine
many different environmental disaster scenarios in real-time, even
with computationally-intensive models. Through this, they will develop
an intuitive understanding of the uncertainty relationships in the
system—from uncertainties in input data through to representations of
uncertainty in model outputs. Although the mathematical and
computational tools we will employ in this project are generalizable
to any modelling environment, our focus is on specific application
domains where (a) the model is (computationally) complex, (b)
uncertainty is significant and (c) decisions are time-sensitive. A key
element to our approach is the use of live coding infrastructure that
we have developed and that we will describe below.

\iffalse
Our vision is that decision makers will be able
to examine many different environmental disaster scenarios in a short
space of time, even with computationally-expensive models. They will 
develop an intuitive understanding of the uncertainty
relationships in the system---from uncertainties in input data through
to representations of uncertainty in model outputs. 
%The importance of
%this human-in-the-loop workflow is expressed by~\cite{pikeScience2009}
%in the following quote: ``it is through the interactive manipulation
%of a visual interface---the analytic discourse---that knowledge is
%constructed, tested, refined and shared.''
Although the mathematical and computational tools we will employ in
this project are generalisable to any modelling workflow, our focus is
on specific application domains where (a) the model is
(computationally) complex, (b) uncertainty is significant and (c)
decisions are time-sensitive. Specifically, we will build interactive
interfaces for riverine floods, storm surge and tsunami modelling,
which exemplify these characteristics.
\fi


\subsubsection*{Background}

{\bf Uncertainty quantification} describes a collection of
mathematical techniques that allow the predictions of computational
simulations to be bounded probabilistically. The area is important in
high-performance-computing (HPC) simulations of scientific and
engineering systems where it is desired to know how likely the
outcomes of a simulation will be when various assumptions underlying
the simulations are open to doubt. When uncertainty is included into a
model, the essential dimensionality of the problem is increased
markedly and sophisticated mathematical techniques are then required
to deal with the “curse of dimensionality” where the cost of
computation increases exponentially with the dimension of the
problem. This project will exploit new mathematical techniques that
combine {\bf reduced order models} (such as sparse grids and reduced
basis methods) with uncertainty quantification. Sparse grids
~\parencite{BungartzGriebel2004} are known to reduce the effects of
the curse of dimensionality and recent work in our group has found new
ways to incorporate gradient information and multi-fidelity models
into sparse grid approximations
~\parencite{deBaarHarding2015,Jakeman2015,deBaarRDM2015}.  Reduced
basis models normally involve the computation of a large number of
`offline' simulations that are examined together with an `online'
simulation to quantify its uncertainty. With our live-coding approach,
we plan to modify this approach so that the bounding simulations are
also  `online' and `live' in a way that allows them to be extended and
interpolated and steered in real time.

\iffalse
When uncertainty is included into a model, the essential
dimensionality of the problem is increased markedly.  New and more
sophisticated techniques are then required to deal with the so called
`curse of dimensionality', where the cost of computation increases
exponentially with the dimension of the problem.  The mathematical
component of this project will be based on new developments combining
reduced order models (such as sparse grids and reduced basis methods)
with uncertainty quantification.  Sparse
grids~\parencite{BungartzGriebel2004} are known to reduce the effects
of the curse of dimensionality and recent work in our group has found
new ways to incorporate gradient information and multi-fidelity models
into sparse grid
approximations~\parencite{deBaarHarding2015,Jakeman2015,deBaarRDM2015}.
We will extend this work to construct efficient techniques for
incorporating uncertainty information into storm surge-tsunami models.
Reduced basis methods~\parencite{quarteroni2015reduced} provide an
alternative approach for constructing reduced order models and will
also be investigated.
\fi

{\bf Live coding} describes software systems that support the direct
intervention of the programmer in a program’s run-time state. It can
be thought of as an extreme version of agile
programming~\parencite{fowlerAgile2001}, where code changes are
hot-swapped into running programs, allowing for extremely fast
exploration and iteration of new ideas and system updates. As the
ambition of live-coding has grown, support systems and languages have
evolved to, for example, create, modify and interact with music and
hardware devices in real-time [refs]. Such an approach has been
described by one of our team members as ``with-time
programming''~\parencite{sorensen2010programming} because it allows
for timing constraints on a running system, including
human-computer-interaction constraints to be explicitly modelled and
guaranteed. The present project will make use of the {\em Extempore
  live-coding software
  environment}\footnote{\url{http://extempore.moso.com.au}} that can
harness and steer scientific simulations (written in ABI compatible
languages such as C, C++ and Fortran) and evaluate (and visualise) the
outcomes of those simulations in real time. Such an intervention in
the world of high-performance computer simulation radically changes
the landscape and ambition of simulation codes; no longer do they need
to be considered as hands-off batch processes running on
supercomputers, but they can now be interacted with on-the-fly while a
simulation is in progress. We envision that there will be several
benefits of the application live-coding to our problem: firstly, it
will enable us to rapidly prototype our simulation software and to
deliver systems which can deliver feedback on a myriad of demands of a
disaster emergency response; secondly, live coding will allow us to
maintain and tune a set of, traditionally-offline simulations used to
quantify uncertainty in real time; and thirdly, the use of live coding
will allow us to adjust and optimise our systems for the volatile HPC
systems expected in disaster management situations as described below.

Computational decision support systems for disaster management have
existed for many decades \parencite{wallaceDecision1985}, and
advances have been made both in incorporating
uncertainty \parencite{thompsonSocial2014,nealeNavigating2015}
and providing real-time interaction and
output~\parencite{yuSupport2006}. More recently, mixed-reality game
scenarios have been used to understand and optimise human-agent
collaboration for disaster response~\parencite{ramchurn2016human}. In
our project plan we will follow a similar approach and will employ
mock-game scenarios to examine and understand the nature of decision
making with expert modelling support. In a novel twist, we will run
these games together with live-coding optimisation and tuning of the
software system and the underlying computational platform. By
analysing the protocols of these live optimisations, we will
accumulate data to feed into redesign of the software interface and to
understand the time requirements and controls needed for the
computational platform.


\iffalse
\newpage
\subsubsection*{Discoveries and Benefits}

This project will provide the following discoveries and benefits:
\begin{enumerate}

\item a suite of interactive, real-time modelling tools for
  surge-tsunami flood disasters which combine high fidelity
  simulations (fine grids) with low fidelity (coarse grid) simulations
  to quantify uncertainty, optimised for human exploration

\item a `live software engineering' approach to the development,
  deployment and optimisation of these interactive software systems

\item models of group interaction scenarios for decision-making with
  expert modelling support; from these models, we will obtain  empirical estimates of
  the time constraints that such scenarios impose on software
  and computer system requirements and the live controls needed to run
  them effectively in volatile contexts

\item interactive information visualisation of simulation predictions
together with uncertainties
\end{enumerate}
\fi


This research is significant because it aims to unlock the power of
sophisticated computational simulation incorporating  uncertainty 
for \emph{interactive} use.
Although we concentrate our research on simulation support for
disaster response, the ultimate potential of this work is to
eventually empower domain experts from a broad range of areas to
better use the high-performance computing power which is now available
to them. We envision a future where performing a complex flood model
or disaster simulation is as interactive and \emph{alive} as flicking
through photos on a tablet.

