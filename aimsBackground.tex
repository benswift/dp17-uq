\subsection*{AIMS AND BACKGROUND}

When systems are disrupted during environmental disasters (for example,
during floods, storm surges or tsunamis) information from
computational simulation is needed urgently to aid decision-making by
emergency services. This is not simply a matter of compute power but
requires insight from applied mathematics, software
engineering, human computer interaction, high performance computing
and data visualisation. All of these facets need to be tightly coupled
and tested in realistic team-coordination and decision-making
scenarios where decision makers collaborate with modelling experts. In
particular, a crucial aspect of advice provided from modellers to
decision makers is to properly quantify and communicate the
\emph{uncertainty} in predictions obtained by computer simulations.
This advice needs to be provided in a timely manner and in a context
where the computer support may vary dramatically over time due to
system outages.

We will address this grand challenge through a \emph{live programming}
approach to the development and deployment of software systems for
environmental simulation in the presence of uncertainty. We will
\textbf{develop} new `inundation models' (models of riverine floods,
storm surges and tsunamis) which use sparse grids and uncertainty
quantification to dramatically increase the speed and usefulness of
model predictions and their associated uncertainty. We will
\textbf{deploy} `live' distributed computing infrastructure to
\emph{rapidly prototype} the human interface to these new
environmental simulations and to transform the traditional,
batch-oriented workflow of environmental simulation into a highly
interactive one. Indeed live programming offers an unprecedented opportunity
for interactive parallel program performance evaluation and tuning.
We will \textbf{evaluate} team coordination and
decision-making in the presence of uncertainty through role-playing
scenarios of representative flood and tsunami situations. In these mock
field trials, we will record (both through instrumented system logs
and interviews with participants) the interaction patterns and
protocols of live programmed optimisation of the software modelling
systems. From a human-computer interaction (HCI) analysis of these
protocols, we will \textbf{identify} patterns in the interaction
states and transitions between those states which we will use to
iterate our software-system development.

Our vision is that using this approach, decision makers will be able
to examine many different environmental disaster scenarios in a short
space of time, even with computationally-expensive models. This will
enable them to develop an intuitive understanding of the uncertainty
relationships in the system---from uncertainties in input data through
to representations of uncertainty in model outputs. The importance of
this human-in-the-loop workflow is expressed by~\cite{pikeScience2009}
in the following quote: ``it is through the interactive manipulation
of a visual interface---the analytic discourse---that knowledge is
constructed, tested, refined and shared.''

Although the mathematical and computational tools we will employ in
this project are generalisable to any modelling workflow, our focus is
on specific application domains where (a) the model is
(computationally) complex, (b) uncertainty is significant and (c)
decisions are time-sensitive. Specifically, we will build interactive
interfaces for riverine floods, storm surge and tsunami modelling,
which exemplify these characteristics.

When uncertainty is included into a model, the 
essential dimensionality of the problem is increased markedly.  
New and more sophisticated techniques are then required to deal with the so called
`curse of dimensionality', where
the cost of computation increases exponentially with the dimension of
the problem. 

The mathematical component of this project will be based on new
developments combining reduced order models 
(such as sparse grids and reduced basis methods)
with uncertainty quantification.
Sparse grids~\parencite{BungartzGriebel2004} are known to
reduce the effects of the curse of dimensionality and 
recent work in our group has found new ways to incorporate gradient
information and multi-fidelity models into sparse grid
approximations~\parencite{deBaarHarding2015,Jakeman2015,deBaarRDM2015}.
We will extend this work to construct efficient techniques 
for incorporating uncertainty  information into 
storm surge-tsunami models. 
Reduced basis 
methods~\parencite{quarteroni2015reduced} provide an 
alternative approach for constructing reduced order 
models and will also be investigated.

Computational decision support systems for disaster management have
existed for many decades \parencite{wallaceDecision1985}, and
advances have been made both in incorporating
uncertainty \parencite{thompsonSocial2014,nealeNavigating2015}
and providing real-time interaction and
output~\parencite{yuSupport2006}. More recently, mixed-reality game
scenarios have been used to understand and optimise human-agent
collaboration for disaster response~\parencite{ramchurn2016human}. In
our project plan we will follow a similar approach and will employ
mock-game scenarios to examine and understand the nature of decision
making with expert modelling support. In a novel twist, we will run
these games together with live-coding optimisation and tuning of the
software system and the underlying computational platform. By
analysing the protocols of these live optimisations, we will
accumulate data to feed into redesign of the software interface and to
understand the time requirements and controls needed for the
computational platform.

\medskip
This project will provide the following discoveries and benefits:
\begin{enumerate}

\item a suite of interactive, real-time modelling tools for
  surge-tsunami flood disasters which combine high fidelity
  simulations (fine grids) with low fidelity (coarse grid) simulations
  to quantify uncertainty, optimised for human exploration

\item a `live software engineering' approach to the development,
  deployment and optimisation of these interactive software systems

\item models of group interaction scenarios for decision-making with
  expert modelling support; from these models, we will obtain  empirical estimates of
  the time constraints that such scenarios impose on software
  and computer system requirements and the live controls needed to run
  them effectively in volatile contexts

\item interactive information visualisation of simulation predictions
together with uncertainties
 

\end{enumerate}

This research is significant because it aims to unlock the power of
sophisticated computational simulation for \emph{interactive} use.
Although we concentrate our research on simulation support for
disaster response, the ultimate potential of this work is to
eventually empower domain experts from a broad range of areas to
better use the high-performance computing power which is now available
to them. We envision a future where performing a complex flood model
or disaster simulation is as interactive and \emph{alive} as flicking
through photos on a tablet.

