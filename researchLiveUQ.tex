\subsubsection*{Live Coding for Uncertainty
  Quantification}

This project will
make use of the \emph{Extempore} live-coding software
environment.%\footnote{\url{http://extempore.moso.com.au}}
 Extempore has already
been used for the live modification and real-time visualisation of
particle-in-cell plasma-physics simulation codes with
negligible performance overhead compared to batch-mode execution in
C~\parencite{swiftLive2016}. 



Such a live deployment of a traditional, batch-oriented HPC simulation allows a user to 
modify the
domain size and shape, the initial and boundary conditions, and various
other parameters of a simulation while that simulation is running. It is of use
for optimising software for later, stand-alone,
deployment as well as for harnessing and steering simulation codes
after deployment.

%The Extempore software environment is a key tool for this project as
%it allows us to fine tune our suite of simulation software for the
%specific requirements of the task domain.

The ability to stop, modify, or restart computations `in flight' has
the potential to significantly improve the efficiency of an
uncertainty analysis. There exist many algorithms, for example
adaptive Markov chain Monte Carlo methods~\parencite{GilksEtal1994}, 
which attempt to choose the best samples based on the sampling
history. For complex problems however, a domain expert
may often have
a better idea about the region of the parameter domain where function
evaluations should be concentrated. Through live programming within a
tight feedback loop a domain expert can incrementally guide the
current sampling strategy being used for uncertainty 
quantification, and in turn be guided by 
real time information derived from the reduced order model 
(such as surpluses
provided by sparse grid approximation to 
identify important parameter dimensions and regions of interest), 
to improve the end result. The resulting strategies are expected to be
more aggressive in nature as they are better targeted to the specific
problem at hand. The result of this should be more efficient quantification of
uncertainty.

In this project, using live coding to accelerate the feedback loop between the vector of input parameters,
$\mathbf{p}$,  and quantities of interest  will give a scientist the ability to interactively
\emph{explore} the connection (and the associated uncertainty) between
the different dimensions of $\mathbf{p}$ and the overall response of
the system. Such a prototyping and optimising of these models ({\bf Aim 2 of our project}) will help us rapidly scope and deliver 
robust models. The ``liveness'' that this approach brings will also be deployed 
into the uncertainty quantification models. For example, reduced-basis methods for uncertainty 
quantification typically employ a number of ``offline'' simulations that are used together with an ``online'' simulation of 
a system of interest to provide uncertainty estimates for that online simulation; we will apply novel live-coding
approaches to such methods where the offline as well as the online simulations are all harnessed and live. The 
potential advantage of such an approach is that the offline simulations can be restarted, modified, reduced and 
explored at the same time that the online simulation is running. Such a research agenda in live coding is one that will
include human-computer interaction as well as software engineering and we will be looking to identify states of interaction 
of expert users with our live-coding systems as well as transitions between them -- much as we have already done so in the 
context of computer music~\parencite{swift2014coding}.






%This will be used to {\bf rapidly prototype software} with a view to 
%dentifying the necessary interactive controls needed to modify and interact with 
%the final, relased and running, software in real-time. 
%By understanding the human-factors and systems-level time
%constraints on the delivery of useful information in disaster-response
%settings, we will also be able to specify timing requirements on our model
%simulations as in the next section. 
%By developing software that is time-constrained and
%time-aware, we will be able to provide decision makers with
%information when it is needed and with an estimate on the uncertainty
%of that information.
%Ultimately, the scientist needs an \textbf{interactive interface} for
%gleaning insights from their models in the presence of these
%challenges.
%\newpage

%\begin{figure}
 % \centering
  %\includegraphics[width=\textwidth]{figures/sg-surrogate-model-fb-loop.pdf}
  %\caption{Using the sparse grids and reduced basis models, the computationally
   % expensive model calculations can be done ahead-of-time and used to
    %construct a surrogate model which can be used to re-claim the
    %interactive workflow of Figure~\ref{fig:unrolled-fb-loop}}.
  %\label{fig:sg-surrogate-model-fb-loop}
%\end{figure}

%Impact: new algorithms, software, high performance computer systems
%and visualisation techniques for time-bound environmental simulation
%with uncertainty. New software tools for the simulation of flood
%surges and tsunami. New methodologies for rapid and agile software
%development and usability using live programming. New knowledge of
%human-in-the-loop requirements for support systems in the context of
%environmental disaster management. New algorithms, software, high
%performance computer systems and visualisation techniques for
%time-bound environmental simulation with uncertainty. New software
%tools for the simulation of flood surges and tsunami. New
%methodologies for rapid and agile software development and usability
%using live programming. New knowledge of human-in-the-loop
%requirements for support systems in the context of environmental
%disaster management.


