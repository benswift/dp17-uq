\documentclass[a4paper,fontsize=12pt]{scrartcl}
\usepackage{Alegreya}
\usepackage{AlegreyaSans}

\usepackage[svgnames,hyperref]{xcolor}
\usepackage{url}
\usepackage{graphicx}

% \usepackage{fullpage}

\usepackage[%
backend=biber,
style=authoryear, % numeric-comp
maxbibnames=10,
url=false,
doi=false]{biblatex}

% \AtEveryBibitem{\clearfield{month}}
% \AtEveryCitekey{\clearfield{month}}

\addbibresource{references.bib}

\usepackage[english=british,threshold=15,thresholdtype=words]{csquotes}
\SetCiteCommand{\parencite}

\newenvironment*{smallquote}
{\quote\small}
{\endquote}
\SetBlockEnvironment{smallquote}

\usepackage[%
unicode=true,
hyperindex=true,
bookmarks=true,
colorlinks=true, % change to false for final
pdfborder=0,
allcolors=DarkBlue,
% plainpages=false,
pdfpagelabels,
hyperfootnotes=true]{hyperref}

\author{}

\date{\today}

\title{Human-in-the-loop Model Uncertainty Quantification for
  Time-Critical Decisionmaking (TBC)}

\begin{document}

\maketitle

\noindent
\textbf{Note to readers:} this is all really rough---just a grab-bag
of related ideas, really---but you've gotta start somewhere.
Comments/edits very welcome.

\section{Introduction}
\label{sec:introduction}

Inverse uncertainty quantification is a critical aspect of any
simulation/modelling workflow. As numerical modelling becomes
increasingly pervasive---from industrial design to basic science to
economic policymaking---the ability to understand and update these
models in response to measurements from the real world is critical in
translating insights from modelling and simulation to actionable
knowledge. A rich understanding of the uncertainty in a model is also
important when the model is used as an input to a human decisionmaking
process (as most models invariably are). What can we say about the
strength of the connection of the different aspects of the model to
reality? How confident can we be about various predictions given by
the model?

In addition, recent developments in sensor networks and pervasive
monitoring technologies have increased our ability to gather data from
the world. Streaming data analysis is an active area of research, and
in an uncertainty quantification context streaming data provides the
opportunity to be continuously estimating model uncertainties and
updating our models in response to new data coming in.

This project asks the question: \textbf{can streaming data be used to
  update and model uncertainties and provide intelligible insights in
  time-critical decisionmaking scenarios?} (TODO this needs to be
tighter)

One example scenario: sensors detect an earthquake in the Pacific Ocean,
and the countdown timer begins---what is the likelihood of a
destructive tsunami, where will it impact, and does an evacuation
order need to be issued? Some initial modelling has already been done,
but new readings are coming in regularly from sensor buoys throughout
the ocean. How can this new (and potentially noisy) data be used to
update the model, and how does this affect the predicted impact? How
does uncertainty in the buoy readings affect our confidence in the
model?

This problem has two main components: a mathematical component, and a
a human factors component.
\begin{itemize}
\item We will deal with the \textbf{mathematical challenges} using a
  sparse grids approach, using combination techniques
  to incorporate new measurements/information in an efficient way.
\item We will deal with the \textbf{human factors challenges} using a
  live steering (maybe even live programming?) approach, building
  interfaces which provide expert decisionmakers with the ability to
  steer computation/optimisation procedures, exploring different model
  parameterisations and scenarios with real-time feedback.
\end{itemize}

That's all I've got. Merry Christmas!

\section{TODO}
\label{sec:todo}

\begin{itemize}
\item references
\item example projects
\item all the rest of it\ldots
\end{itemize}

% \section{Conclusion}
% \label{sec:conclusion}

% Please give us all the money, and we'll do amazing science.

\printbibliography[title=References]

\end{document}
