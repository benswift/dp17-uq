\subsubsection*{Project organisation}

The two postdoctoral fellows and four of the five PhD students 
will be concerned with the main aims of this grant application: One postdoctoral fellow and two PhD students in 
Mathematics will study storm surge equations in realistic topographies together with the calculation of uncertainty. They will work closely with the other postdoctoral fellow 
and one PhD student in Computer Science to study the application of live coding to uncertainty quantification of these equations. The fourth
PhD student will work with the postdoctoral fellow in Computer Science to study live coding of the HPC infrastructure for these systems.

The remaining PhD student will study the human-computer-interaction context of real-time disaster management. 
Similar to the impact of visually presented geodata on decision
making~\parencite{kinkeldey2015evaluating}, the particular
visualisations of uncertainty in our environmental models will need to
be systematically evaluated with human participants. Here we will
adopt traditional human-factors trials with non-expert participants
together with qualitative feedback from emergency services experts. We
will solicit qualitative feedback from expert participants from
emergency authorities in our local area including ACT Emergency
Services, Geosciences Australia, the Australian Maritime and Safety
Authority and the Australian Federal Police. Perspectives offered by
this participant pool will be important in extrapolating our study
results to real world disaster-management.

We anticipate a number of Honours and other student projects concerned with the overall context of our research. The estimate 
of 8 Honours projects in our application is on the low side of this number; there are many 
fascinating problems in developing robust simulations of envinromental disasters
within the command and control context of emergency management to 
keep large numbers of students from mathematics and computer science gainfully employed over the three years of this grant.






 



