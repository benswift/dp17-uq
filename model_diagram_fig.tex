% Brendan: a draft diagram based on one of the TikZ examples

\tikzstyle{block} = [draw, fill=blue!20, rectangle, 
    minimum height=3em, minimum width=6em]
\tikzstyle{sum} = [draw, fill=blue!20, circle, node distance=1cm]
\tikzstyle{input} = [coordinate]
\tikzstyle{output} = [coordinate]
%\tikzstyle{pinstyle} = [pin edge={to-,thin,black}]

% The block diagram code is probably more verbose than necessary
\begin{tikzpicture}[auto, node distance=2cm,>=latex']
    % We start by placing the blocks
    \node [input, name=input] {};
    \node [sum, right of=input, node distance=2cm] (sum) {};
    \node [block, right of=sum, node distance=3cm] (model) {Model $M_{\mathbf{p}}$};
    \node [block, right of=model, %pin={[pinstyle]above:Disturbances},
            node distance=4cm] (processing) {Post-Processing};
    % We draw an edge between the controller and system block to 
    % calculate the coordinate u. We need it to place the measurement block. 
    \draw [->] (model) -- node[name=u] {$u_{\mathbf{p}}$} (processing);
    \node [output, right of=processing, node distance=3cm] (output) {};
    \node [block, below of=u] (update) {Update model parameters};

    % Once the nodes are placed, connecting them is easy. 
    \draw [draw,->] (input) -- node {initial $\mathbf{p}$} (sum);
    \draw [->] (sum) -- node {$\mathbf{p}$} (model);
    \draw [->] (processing) -- node [name=y] {$Q(u_{\mathbf{p}})$}(output);
    \draw [->] (y) |- (update);
    \draw [->] (update) -| node[pos=0.99] {$+$} 
        node [near end] {$\Delta\mathbf{p}$} (sum);
\end{tikzpicture}